% !TEX root = 7350.tex




\section{Introduction}

Conrad's note in \cite{aws-2008} is a good overview. For a long term, 
\cite{bgr-1984} was the main reference, but it is difficult to read. We will 
primarily use \cite{berkovich-1990} and \cite{bosch-2014} as more technical 
references. 

We'll partially follow \cite{berkovich-1990}, which is freely available on 
the Cornell Library. The algebraic and analytic foundations for this course 
are developed in \cite{bgr-1984}. 

The course will largely be self-contained, but we'll occasionally refer to 
outside theorems. The subject is currently very active; a lot of foundational 
material is not contained in books. 

Non-archimedean geometry is related to many different fields. It uses 
algebraic geometry, analysis, algebra, and has applications in those fields, 
combinatorics, and number theory. Many of the rings that arise are 
non-noetherian, for example. If this course lived ``over $\bC$,'' it would 
be a combination of functional analysis, commutative algebra, and Riemann 
surfaces. 

Non-archimedean geometry has many fantastic application (local Langlands, etc.) 
but sadly, we will not be able to say much about contemporary applications. 





\subsection{Motivation and background}

First, what is the archimedean property? Fix a field $K$, and a ``norm'' 
(we'll give precise definitions later) $|\cdot|:K\to \bR$. We say that $K$ 
satisfies the \emph{archimedean property} if for any $x\in K^\times$, there 
exists $n\in \bZ$ such that $|n x|>1$. 

\begin{example}
The fields of real and complex numbers, with their classical absolute 
values, are archimedean.
\end{example}

According to a theorem of Gelfand and Mazur, $\bR$ and $\bC$ are the ``only'' 
examples of archimedean fields. Since there are many examples of 
non-archimedean fields, there is a sense in which $\bR$ and $\bC$ are 
exceptional. 

\begin{example}
In classical (complex) algebraic geometry, one is interested in the zero sets 
of polynomials in $\bC^n$. Call such sets \emph{algebraic}. This gives us a 
topology on $\bC^n$ and its subvarieties known as the \emph{Zariski topology}. 
We can also give $\bC^n$ the topology coming from the absolute value $|\cdot|$ 
on $\bC$. This finer topology is called the \emph{canonical topology}. It 
allows us to do analysis on complex varieties. 
\end{example}

The canonical topology (and the analytic techniques accompanying it) allow for: 
\begin{itemize}
  \item Cauchy integrals
  \item holomorphic / meromorphic functions / differential forms
  \item Hodge theory
  \item Morse theory
\end{itemize}
and there are formal GAGA (for \emph{\textbf{G}\'eom\'etrie \textbf{A}lg\'ebrique et \textbf{G}\'eom\'etrie 
\textbf{A}nalytique}) corresponces after Serre's paper \cite{serre-1956}. Essentially, Serre's 
paper says that as far as sheaves and cohomology on proper varieties are 
concerned, algebraic geometry is the same as analytic geometry. 

In modern algebraic geometry, we also care about solutions (to polynomial 
equations) in $K^n$, where $K$ is for example $\bQ$, $\bQ_p$ (in number 
theory), or $\bC\laurent{t}$ (in the deformation theory of complex varieties). 
All these fields have interesting absolute values which are \emph{not} 
archimedean.

\begin{example}
Any field $K$ can be given the ``stupid'' absolute value by 
\[
  |x| = \begin{cases} 1 & x\ne 0 \\ 0 & x=0 \end{cases} .
\]
Although it seems like a degenerate case, the theory of the ``stupid'' 
absolute value is highly interesting. 
\end{example}

\begin{example}
The field $K=\bQ$ has two classes of absolute values. One is the classical 
absolute value, which we denote by $|\cdot|_\infty$. There is of course the 
``stupid'' absolute value, which we denote $|\cdot|_0$. Fix a prime $p$. There 
is an absolute value $|\cdot|_p$ on $\bQ$, defined by 
\[
  \left|\frac a b\right|_p = \left|p^e \frac{a'}{b'}\right|_p = p^{-\alpha} ,
\]
where $\frac a b=p^e \frac{a'}{b'}$ is such that both $a$ and $b$ are 
relatively prime to $p$. 
\end{example}

In a precise sense, Otrowski's theorem tells us that 
$\{|\cdot|_\infty,|\cdot|_0,|\cdot|_p:\text{$p$ prime}\}$ is essentially a 
complete list of the possible absolute values on $\bQ$. Note that 
$|\cdot|_p$ and $|\cdot|_0$ are non-archimedean. Equivalently, the familiar 
triangle inequality $|x+y|\leqslant |x|+|y|$ can be strengthened to the 
inequality 
\[
  |x+y| \leqslant \max\{|x|,|y|\} .
\]
It is easy to prove that this ``strong triangle inequality'' holds for 
$|\cdot|_0$ and the $|\cdot|_p$. 

It's a good idea to complete $\bQ$ with respect to $|\cdot|_p$. We write 
$\bQ_p$ for this completion, called the field of \emph{$p$-adic numbers}. 
(Incidentally, it is easy to check that $\bQ$ is not complete with respect 
to $|\cdot|_p$. Alternatively: general theorem about complete metric spaces\ldots). 

\begin{example}
The field $\bC\laurent{t}$ is often given the \emph{$t$-adic} absolute 
value: 
\[
  \bigg|\sum_{i\geqslant j} a_i t^i\bigg|_t = e^{-j} ,
\]
whenever $a_k\ne 0$. The algebraic closure of $\bC\laurent{t}$ has an easy 
explicit description, and is called the field of \emph{Puiseaux series}, first 
studied by Isaac Newton. 
\end{example}

Recall how one constructs $\bR$ from $\bQ$. First form the ring of Cauchy 
sequences, then mod out by the (maximal) ideal of sequences converging to zero. 
The same procedure works for any topological field. However, when we completed 
$\bQ$ to get $\bR$, the operation of ``filling in the holes'' turned the 
totally disconnected $\bQ$ into a connected space. This phenomenon is 
pathological! In the non-archimedean world, the completion is still totally 
disconnected. In fact, \emph{all} non-archimedean fields are totally 
disconnected. This would seem to make any kind of analysis difficult (for 
example: how to discuss paths?). 

It makes sense to talk about power series and their convergence over 
non-archimedean fields. Indeed, convergence of formal power series is far 
easier to check in the non-archimedean case than otherwise! (A series converges 
if and only if its terms tend to zero.) Recall that in complex analysis, 
a function is analytic if it can locally be written as a convergent power 
series. For $X$ some ``geometric object'' over a non-archimedean field $K$, we 
can define a function $f\colon X\to K$ to be analytic if it is 
locally presentable by a convergent power series. This is a bad definition 
because the notion of ``locally'' behaves badly for totally disconnected 
spaces. 





\subsection{Approaches to non-archimedean geometry}

Before the 1960's, things stood here; no satisfactory theory existed. Today 
a plethora of solutions to this problem exist:

\subsubsection{Tate-Grothendieck, 1960's}
John Tate realized that certain types of 
elliptic curves can be ``formally uniformized'' over arbitrary fields. He wrote 
to Grothendieck with this discovery, but Grothendieck was unimpressed. Despite 
this, Tate was able to use Grothendieck's mathematical machinery to construct 
the category of \emph{rigid analytic spaces}, in which uniformization makes 
sense. Better theories exist today, but many foundational results are only 
written in the language of rigid spaces. The main difficulty here is that the 
``spaces'' involved are not actually topological spaces; they only carry a 
Grothendieck topology. The introduction to \cite{aws-2008} has a good 
historical overview. Affinoid algebras, the analytic substitute for polynomial 
rings, are the only aspect of the theory still actively used today. 

\subsubsection{Raynaud, 1970's}
Raynaud, a former student of Grothendieck, developed an extremely 
powerful technique for approaching non-archimedean geometry using ``formal 
models.'' A downside is that the theory is quite technical. It is analogous to 
writing down a variety $V_{/\bQ}$ as the zero-set of polynomials with 
coefficients in $\bZ$. In Raynaud's theory, the ``model'' for a rigid space is 
a formal scheme. A non-archimedean field $K$ contains a valuation ring $R$, and 
one represents rigid $K$-varieties with formal $R$-schemes. Raynaud's theory is 
good for answering ``algebraic'' questions, e.g.~of flatness, base change, 
fiber dimension, \ldots. 

\subsubsection{Berkovich, early 1990's}
Berkovich spaces, developed in \cite{berkovich-1990}. This turns Tate's 
rigid spaces into honest topological spaces by adding points. One ends up with 
a topological space together with a structure sheaf. The topological space is 
compact, Hausdorff, and locally path connected on connected components. In a 
precise sense, a Berkovich space is a ``space of rank-one valuations.'' 

\subsubsection{Huber, late 1990's}
Huber has defined in \cite{huber-1996} a category of adic spaces. In a 
precise sense, an adic 
space is the ``limit of all formal models'' (Riemann-Zariski space). 
Equivalently, an adic space is the ``space of all valuations'' on some 
reasonable ring. Recently, Scholze used adic spaces in \cite{scholze-2012} to prove 
spectacular theorems. 


Let $K$ be a non-archimedean field. Berkovich's idea is to use seminorms 
to ``add points'' to $K$, to result in a $\bR$-tree. This construction 
can be glued to turn each variety $X_{/K}$ into a topological space 
$X^\an$. It is an extremely 
hard theorem that this ``Berkovich space'' deformation retracts onto a 
finite simplicial complex. 

\begin{theorem}[Berkovich, Thuillier, Loeser-Hrushovski]
Any analytic space (in the sense of Berkovich) has a strong deformation retract 
onto a finite simplicial complex. 
\end{theorem}

See \cite{berkovich-1999,thuillier-2007} for partial results in this direction. 
So the ``wild'' spaces we will construct will always be homotopy equivalent to 
something managable. For curves, this simplicial complex will be a graph. 





\subsection{Tropicalization}

Consider the curve $X\colon z_1+z_2=1$. The idea of behind \emph{am\oe{}bas} is that 
instead of looking at $X(\bC)$, which is hard to visualize, one should consider the 
set 
\[
  \trop_t(X) = \{(-\log_t|z_1|,-\log_t|z_2|)\in \bR^2\colon (z_1,z_2)\in X(\bC)\} .
\]
As a subset of $\bR^2$, this can be easily plotted. If we let $t\to 0$, then 
the limiting set exists; it is a union of line segments. The better approach is 
to consider the following set: 
\[
  \trop(X)=\{(v_t(z_1),v_t(z_2))\colon (z_1,z_2)\in X(\bC\laurent t)\} .
\]

More generally, let $K$ be a non-archimedean field, $X_{/K}$ a variety. 
Given a rational map $g\colon X\dashrightarrow \Gm^N$, let 
$f\colon X(K)\to \bR^N$ be the map 
\[
  f(x) = (-\log|g_1(x)|,\dots,-\log|g_N(x)|) 
\]
and put $\trop(X) = \overline{\image(f)}$. Clearly this depends on the choice 
of rational map $g$. The tropicalization map factors through the 
analytification of $X$: 
\[
\begin{tikzcd}
  X \ar[r] \ar[dr, dotted]
    & X^\an \ar[d] \\
  & \trop(X)
\end{tikzcd}
\]
So $\trop(X)$ is a ``snapshot'' of $X^\an$. 

\begin{theorem}[Payne]\label{thm:limit-tropicalizations}
$X^\mathrm{an}$ is the limit of all tropicalizations of $X$. 
\end{theorem}

See \cite{payne-2009} for a rigorous formulation and proof. 





\subsection{Applications}

Non-archimedean geometry has many applications in diverse fields. These 
include: 

Semistable reduction theorems in arithmetic geometry. N\'eron models, 
local heights, $p$-adic Hodge theory, Local Langlands. Motivic zeta 
functions, uniformizations, counting curves, arithmetic dynamics, 
complex dynamics, Bogomolov conjecture, minimal model program, 
toroidal embeddings / toric varieties. (Bruhat-Tits) buildings, resolution 
of singularities in characteristic $p$. 




