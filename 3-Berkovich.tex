% !TEX root = 7350.tex

\chapter{Berkovich spaces}

This generalizes Gelfand's theory. Much of the theory even works for 
non-commutative algebras. We do not require the existence of a base field, so 
our theory applies to ``arithmetic'' rings like $\bZ$ or 
$\bZ_2[x]/(7 x^3-5)$. 





\section{Berkovich spectrum of a normed ring}

We could phrase everything in terms of (rank-one) valuations and semivaluations.
Let $A$ be a commutative (unital) ring. 


\subsection{Norms and seminorms}

\begin{definition}
A \emph{seminorm} on $A$ is a function $\|\cdot\|\colon A\to \bR^{\geqslant 0}$ 
satisfying 
\begin{enumerate}
\item
$\|0\|=0$, $\|1\|\leqslant 1$, 

\item
$\|f+g\|\leqslant \|f\|+\|g\|$, 

\item
$\|f g\| \leqslant \|f\| \|g\|$. 
\end{enumerate}
\end{definition}

If we replace axiom 2 with $\|f+g\| \leqslant \max\{\|f\|,\|g\|\}$, we say the 
norm $\|\cdot\|$ is non-archimedean. We will focus primarily on non-archimedean 
norms. We allow $x\ne 0$ to have $\|x\|=0$. Define 
\[
  \ker(\|\cdot\|) = \{x\in A\colon \|x\| = 0\} .
\]
(Huber calls this the \emph{support} of $\|\cdot\|$.) This is a prime ideal in 
$A$. We call $\|\cdot\|$ a \emph{norm} if $\ker(\|\cdot\|)=0$. It turns out 
that our hypotheses imply that either $\|1\|=0$ or $\|1\|=1$. For, if 
$\|1\|\ne 0$, then $\|1\| \leqslant \|1\|\|1\|$, so $1\leqslant \|1\|$. If 
$\|1\|=0$, then $\|f\|=0$ for all $f\in A$. We want to allow this possibility 
so that the zero ring can have a norm. 

If $(A,\|\cdot\|)$ is a ring with seminorm, then $(A,+)$ is naturally a 
seminormed group, i.e.~it satisfies 
\begin{align*}
  \|0\| &=0 \\
  \|f+g\| &\leqslant \|f\|+\|g\| \\
  \|-f\| &= \|f\| .
\end{align*}
(Alternatively, replace the last two equations by 
$\|f-g\| \leqslant \|f\|+\|g\|$.) Any seminormed group carries a natural 
topology, so we get a topology on $A$. This topology is Hausdorff if and only 
if $\ker(\|\cdot\|)=0$. The (Hausdorff) completion $\widehat A$ exists (it can 
be defined via Cauchy sequences as usual), but the obvious map 
$Ai\colon\to \widehat A$ is injective if and only if $\ker(\|\cdot\|)=0$. In 
fact, $|ker(i)=\ker(\|\cdot\|)$. 

If we replace the axiom $\|f g\| \leqslant \|f\| \|g\|$ with 
$\|f g\| = \|f\| \|g\|$ and also assume $\|\cdot\|\ne 0$, we say that 
$\|\cdot\|$ is \emph{multiplicative}. If for all $f\in A$, $\|f^n\| = \|f\|^n$, 
we say $\|\cdot\|$ is \emph{power-multiplicative}. 

If $N$ is a subgroup of a seminormed group $(A,+)$, then we get a ``residue 
seminorm'' on $A/N$, defined by 
\[
  \|g+N\| = \inf\{\|f\|\colon f-g\in N\} .
\]
This gives a norm on $A/N$ if and only if $N$ is closed. 

Let $\varphi\colon M\to N$ be a homomorphism between seminormed groups. We call 
$\varphi$ \emph{bounded} if there exists some $C>0$ such that 
$\|\varphi(f)\|\leqslant C \|f\|$ for all $f\in M$. Clearly bounded functions 
are continuous. The converse does not hold in this level of generality. 

We say two seminorms $\|\cdot\|,\|\cdot\|'$ on a group are \emph{equivalent} if 
there exists $C_1,C_2>0$ such that 
\[
  C_1 \|\cdot\| \leqslant \|\cdot\|' \leqslant C_2 \|\cdot\| .
\]
Equivalent seminorms induce the same topology, but the converse is false in 
general. 

\begin{example}
Let $K$ be any field; consider $M=k[x]$. Let $f=\sum_0^n a_i x^i\in M$, and 
pick some $0<\alpha<1<\alpha'$. Define two seminorms on $M$ by:
\begin{align*}
  \|f\| &= \max\{\alpha^j\colon \alpha_j\ne 0\} \\
  \|f\|' &= \max\{{\alpha'}^j\colon \alpha_j\ne 0\} .
\end{align*}
Let $\varphi\colon (M,\|\cdot\|')\to (M,\|\cdot\|)$ be the identity on $M$. It 
is easy to check that $\varphi$ is bounded. Indeed, 
\[
  \frac{\|\varphi(f)\|}{\|f\|'} = \frac{\|f\|}{\|f\|'} = \frac{\alpha^j}{{\alpha'}^i} \leqslant 1 .
\]
Now $\varphi^{-1}$ is \emph{not} bounded (this is easy), but 
$\varphi^{-1}$ is continuous. We conclude that continuity does not imply 
boundedness. Moreover, bounded homeomorphisms need not have bounded inverses. 
\end{example}

Let $\varphi\colon M\to N$ be a linear map of seminormed groups. As groups, 
$M/\ker\varphi\simeq \image\varphi$, but this isomorphis might not respect the 
seminorms on the two groups. 

\begin{definition}
A linear map $\varphi\colon M\to N$ of seminormed groups is \emph{admissible} 
(or \emph{strict}) if the residue seminorm on $M/\ker\varphi$ and the seminorm 
on $\image\varphi$ are equivalent via the natural isomorphism 
$M/\ker\varphi\simeq \image\varphi$. 
\end{definition}

\begin{definition}
A \emph{Banach ring} is a normed ring which is complete (with respect to the 
norm). 
\end{definition}


\subsection{Examples and first properties}

\begin{example}
Any ring is with the trivial norm 
\[
  \|f\| = \begin{cases} 1 & \text{if }f\ne 0 \\ 0 & \text{if }f=0 \end{cases}
\]
is a Banach ring. 
\end{example}

\begin{example}
The ring $(\bZ,\|\cdot\|_\infty)$ is a Banach ring. The induced topology is 
discrete, but $\|\cdot\|_\infty$ is very far from being equivalent to the 
trivial norm on $\bZ$. 
\end{example}

\begin{example}
Let $A$ be a Banach ring, $\fa\subset A$ a closed ideal. Then $A/\fa$, with the 
residue norm, is a Banach ring. 
\end{example}

\begin{example}
In the context of the above example, if $\fm\subset A$ is any maximal ideal, 
then $A/\fm$ is a Banach field. Here we implicitly take advantage of 
\autoref{lem:maximal-ideal-closed}. 
\end{example}

\begin{lemma}\label{lem:maximal-ideal-closed}
Let $A$ be a Banach ring, $\fm\subset A$ a maximal ideal. Then $\fm$ is closed. 
\end{lemma}
\begin{proof}
If $\fa\subset A$ is any ideal, then its closure $\closure(\fa)$ is also an 
ideal. Of course, if $\fa$ is a proper ideal, then 
$\fa\cap A^\times=\varnothing$. But $A^\times$ is open. Indeed, if 
$x\in A^\times$, then $\disk^-(x,r)\subset A^\times$ for 
$r=\frac 1 2 \|x^{-1}\|^{-1}$. To see this, let $x+h\in \disk^-(x,r)$; we wish 
to show that $x+h\in A^\times$. We know that $\|h\|<\frac 1 2 \|x^{-1}\|^{-1}$, 
so $\|h x^{-1}\| \leqslant \|h\| \|x^{-1}\| < \frac 1 2$. Write 
$x+h=x(1+h x^{-1})$; the element $1+h x^{-1}$ is invertible by 
\autoref{lem:principle-units-invertible}
\end{proof}

\begin{lemma}\label{lem:principle-units-invertible}
Let $A$ be a Banach ring. Then $\disk^-(1,1)\subset A^\times$. 
\end{lemma}
\begin{proof}
We need to show that if $\|x\|<1$, then $1-x$ is invertible. The sequence 
\[
  a_n = 1 + x + \cdots + x^n ,
\]
is Cauchy because $\|x\|<1$, so it converges. One can easily check that the 
limit is $(1-x)^{-1}$. 
\end{proof}

\begin{example}\label{ex:Tate-algebra}
Let $A$ be a Banach ring with norm $\|\cdot\|$. Choose $r\in \bR^{>0}$. We 
define the ring $A\respow{r^{-1} T}\subset A\pow{T}$ to be the set of power 
series $f=\sum_{i\geqslant 0} a_i T^i$ such that 
$\sum \|a_i\| r^i< \infty$. If $A$ is non-archimedean, we only need to require 
$\|a_i\| r^i \to 0$. We call $A\respow{r^{-1} T}$ a ring of \emph{convergent 
power series}. We define a norm on $A\respow{r^{-1} T}$ by: 
\begin{align*}
  \|f\| &= \sum \|a_i\| r^i && \text{archimedean case} \\
  \|f\| &= \max\{ \|a_i\| r^i\} && \text{non-archimedean case} .
\end{align*}
We could have used the first definition in both cases, but the latter is easier 
to handle. Claim: $(A\respow{r^{-1} T},\|\cdot\|)$ is a Banach ring. 
\end{example}

\begin{example}
Let $\{A_i\}_{i\in I}$ be an arbitrary collection of Banach rings. Then the 
direct product $A=\prod A_i$ can be ``too large'' to be a Banach ring. It 
contains a natural Banach ring, namely the \emph{bounded direct product} 
\[
  \prod^\mathrm{b} A_i = \left\{(f_i)_{i\in I}\colon \sup\|f_i\|<\infty\right\} .
\]
This is a banach ring with respect to the norm 
$\|(f_i)\| = \sup \|f_i\|$. Also, there is the \emph{restricted direct 
product}:
\[
  \prod^\mathrm{c} A_i = \left\{(f_i)_{i\in I}\colon \lim \|f_i\|\to 0\right\} ,
\]
and the direct sum $\bigoplus_i A_i$. Note that the restricted direct product 
and direct sum will not be unital rings. 
\end{example}





\subsection{Spectrum: definition and first properties}

Let $(A,\|\cdot\|)$ be a normed ring. A seminorm $|\cdot|$ on $A$ is called 
\emph{bounded} (with respect to $\|\cdot\|$) if there exists $c>0$ such that 
$|f|\leqslant c\|f\|$ for all $f\in A$. One also calls bounded seminorms 
\emph{admissible}. If $|\cdot|$ is bounded, then the induced map 
$|\cdot|\colon A\to \bR,f\mapsto |f|$ is continuous, where $\bR$ has its usual 
topology and $A$ has the topology induced by $\|\cdot\|$. 

\begin{definition}
Let $A$ be a commutative unital Banach ring with fixed norm $\|\cdot\|$. The 
\emph{spectrum} of $A$, denoted $\Berkspec(A)$, is the set of bounded 
multiplicative seminorms on $A$. For $x\in \Berkspec(A)$, let $|\cdot|_x$ be 
the corresponding seminorm. We give $\Berkspec(A)$ the weakest topology making 
all maps $x\mapsto |f|_x$ continuous. 
\end{definition}

\begin{theorem}\label{thm:Berkspec-not-empty}
Let $A$ be a commutative Banach ring. Then 
\begin{enumerate}
\item
$\Berkspec(A)\ne\varnothing$. 

\item
$\Berkspec(A)$ is Hausdorff. 

\item
$\Berkspec(A)$ is compact. 
\end{enumerate}
\end{theorem}
\begin{proof}
1. This is the hardest part of the proof. Let $\fm\subset A$ be a maximal 
ideal; $\fm$ is closed by \autoref{lem:maximal-ideal-closed}. The quotient 
$A/\fm$ is a field, complete with respect to the residue norm. If 
$\Berkspec(A/\fm)\ne\varnothing$, then $\Berkspec(A)\ne\varnothing$. Indeed, 
if $|\cdot|$ is a bounded multiplicative seminorm on $A/\fm$, then 
$|\cdot|\colon A\to \bR$ given by $f\mapsto |f+\fm|$ is a bounded 
multiplicative seminorm on $A$. (This is a special case of functoriality: if 
$f\colon A\to B$ is a bounded homomorphism of Banach rings, there is a natural 
map $f^\ast\colon \Berkspec(B)\to \Berkspec(A)$.) Without loss of generality, 
we may assume that $A$ is a field. 

Let $S$ be the set of all non-zero bounded seminorms on $A$. Given 
$|\cdot|,|\cdot|'\in S$, we put $|\cdot|\leqslant |\cdot|'$ if 
$|f|\leqslant |f|'$ for all $f\in A$. This gives $S$ a partial order. Since 
$\|\cdot\|\in S$, the set $S$ is non-empty. If 
$\{|\cdot|_i\}_{i\in I}\subset S$ is a chain, then $\inf_i |\cdot|_i\in S$ is 
a lower bound for $\{|\cdot|_i\}_{i\in I}$. Zorn's lemma gives us a minimal 
element $|\cdot|\in S$. We claim that 
$|\cdot|$ is multiplicative. To show this, it's enough to show that 
$|f^{-1}| = |f|^{-1}$ for all $f\in A^\times$. For, we already know that 
$|f g|\leqslant |f| |g|$, so just compute:
\[
  |f g|^{-1} = |f^{-1} g^{-1}| \leqslant |f^{-1} |g^{-1}| = |f|^{-1} |g|^{-1} .
\]
We always have $1\leqslant |f| |f^{-1}|$, so assume there exists $f$ with 
$|f|^{-1}<|f^{-1}|$. Let $r=|f^{-1}|^{-1}<|f|$. Consider the 
Tate algebra (from \autoref{ex:Tate-algebra}):
\[
  B = A\respow{r^{-1} T} = \left\{\sum_{i\geqslant 0} a_i T^i\colon \sum |a_i| r^i < \infty\right\} .
\]
Note that $|\cdot|$ extends to $B$ by $|\sum a_i T^i| = \sum |a_i| r^i$. We 
claim that $f-T$ is not invertible in $B$. If it did have an inverse, this 
would be $\sum f^{-i} T^i$, which has norm $\sum |f^{-i}| r^i = \sum 1$, which 
diverges. (See the proof of \cite[1.2.1]{berkovich-1990} for details). Consider 
the map $A\to B/\langle f-T\rangle\ne\varnothing$. This induces a smaller 
bounded seminorm on $A$, which contradicts the minimality of $|\cdot|$. So 
$|\cdot|$ is multiplicative. 

2. For $x\ne y\in \Berkspec(A)$, we want to show there are disjoint open 
neighborhoods of $x$ and $y$. Since $x\ne y$, there exists some $f\in A$ such 
that $|f|_x\ne |f|_y$, say $|f|_x<|f|_y$. Choose $r\in \bR$ with 
$|f|_x<r<|f|_y$. Then $\{z\in \Berkspec(A)\colon |f|_z<r\}$ and 
$\{z\in \Berkspec(A)\colon |f|_z>r\}$ are the desired sets. 

3. Let $C=\prod_{f\in A} [0,\|f\|]$; this is compact by Tychonoff's theorem. 
Note that $\Berkspec(A)\hookrightarrow C$ via $x\mapsto (|f|_x)_{f\in A}$. 
Moreover, $\Berkspec(A)\subset C$ is closed because all the defining properties 
of $\Berkspec(A)$ are closed conditions. Thus $\Berkspec(A)$ is itself 
compact. 
\end{proof}

If $\|\cdot\|$ is already multiplicative, then $\Berkspec(A)\ne\varnothing$ 
because $\|\cdot\|\in \Berkspec(A)$. This is much harder in general. We didn't 
actually need the completeness of $A$ in the above definition. This is because 
any bounded seminorm on $A$ has a unique extension to $\widehat A$. We could 
have replaced $\|\cdot\|$ with any equivalent seminorm and get the same 
$\Berkspec(A)$. 

Let $|\cdot|\colon A\to \bR$ be a power-multiplicative bounded seminorm. Then 
$|f|^n \leqslant c\|f\|^n$; take $n$-th roots and let $n\to \infty$ to realize 
that we may assume $c=1$. So all $x\in \Berkspec(A)$ satisfy 
$|f|_x\leqslant \|f\|$ for all $f\in A$. 

The topology of $\Berkspec(A)$ is very mysterious. In general, one needs model 
theory to prove some basic facts (e.g.~that $\Berkspec(A)$ ``looks like'' a 
simplicial complex). Here are two equivalent ways to define the topology:
\begin{enumerate}
\item
The topology on $X=\Berkspec(A)$ is generated by open sets of the form 
\[
  \{x\in X\colon |f|_x<\alpha\} , \qquad \{x\in X\colon |f|_x>\alpha\} ,
\]
for $f\in A$, $\alpha\in \bR$. 

\item
A filter $\fF$ converges to $x\in X$ if and only if the filter 
\[
  |f|_\fF = \left\{\{|f|_u\colon u\in U\}\colon U\in \fF\right\} 
\]
converges to $|f|_x$ for all $f\in A$.  
\end{enumerate}

Let $x\in \Berkspec(A)$. Then $\fp_x = \ker(|\cdot|_x)$ is a closed prime 
ideal. For $f\in A$, $|f|_x$ depends only on $\bar f\in A/\fp_x$. But 
$|\cdot|_x$ is a bounded multiplicative norm on $A/\fp_x$. The ring $A/\fp_x$ 
is an integral domain, so we can pass to its field of fractions 
$(A/\fp_x)_{(0)}$, which carries the norm induced by $|\cdot|_x$. Let 
$\sH(x) = \widehat{(A/\fp_x)_{(0)}}$; this is the \emph{completed residue 
field} of $A$ at $x$. For $f\in A$, write $f(x)$ for the image of $f$ under the 
composite map 
\[
  A \twoheadrightarrow A/\fp_x \hookrightarrow (A/\fp_x)_{(0)} \hookrightarrow \widehat{(A/\fp_x)_{(0)}} = \sH(x) .
\]
We will write $|\cdot|$ for the canonical extension of $|\cdot|_x$ to 
$\sH(x)$. So $|f(x)| = |f(x)|_x = |f|_x$. The map $A\to \sH(x)$ is a 
``character,'' in the following sense. 

\begin{definition}
Let $A$ be a Banach ring. A \emph{character} on $A$ is a nonzero bounded 
homomorphism to a field complete with respect to some absolute value. 
\end{definition}

\begin{definition}
Let $A$ be a Banach ring. The \emph{Gelfand transform} on $A$ is the natural 
map 
\[
  A\to \prod^\mathrm{b}_{x\in \Berkspec(A)}  \sH(x) \qquad f\mapsto (f(x))_{x\in \Berkspec(A)} .
\]
\end{definition}

Let $B=\prod_{x\in \Berkspec(A)}^\mathrm{b} \sH(x)$. Then $B$ is a Banach ring, 
and the Gelfand transform $A\to B$ is a bounded map. We get an induced 
surjective map $\Berkspec(B)\to \Berkspec(A)$. 

\begin{lemma}
Let $A$ be a Banach ring. Then $f\in A$ is invertible if and only if 
$f(x)\ne 0$ for all $x\in \Berkspec(A)$. 
\end{lemma}
\begin{proof}
If $f\in A^\times$, then $|1|_x = |f f^{-1}|_x = 1$, so $|f|_x\ne 0$, whence 
$f(x)\ne 0$ in $\sH(x)$ for all $x$. If $f\notin A^\times$, then there is some 
maximal (hence prime) ideal $\fm\ni f$. By \autoref{thm:Berkspec-not-empty}, 
$\Berkspec(A/\fm)\ne\varnothing$. Choose $|\cdot|\in \Berkspec(A/\fm)$, and 
let $|\cdot|_x\in \Berkspec(A)$ be its pullback. Then 
$|f|_x = |f+\fm| = |\fm| = 0$. 
\end{proof}

Let $x\in \Berkspec(A)$. We have seen that there is a natural character 
$\chi_x\colon A\to \sH(x)$. Conversely, given a character $\chi\colon A\to K$, 
we get a bounded multiplicative seminorm $|\cdot|_\chi\colon A\to \bR$, defined 
by $|f|_\chi = |\chi(f)|$. At the moment, different characters might induce the 
same seminorm on $A$. We handle this by noting that two characters 
$\chi',\chi''$ give the same point in $\Berkspec(A)$ if and only if they are 
equivalent in the following sense: there exists a character $\chi\colon A\to K$ 
such that the following diagram commutes:
\begin{equation}\label{eq:equiv-characters}
\begin{tikzcd}
  & A \ar[dl, swap, "\chi'"] \ar[d, "\chi"] \ar[dr, "\chi''"] \\
  K' 
    & K \ar[l, hook] \ar[r, hook]
    & K''
\end{tikzcd}
\end{equation}


\subsection{Comparison with algebraic spectrum}

Let $A$ be a (commutative, unital) ring. Recall that $\spec(A)$ is the set of 
prime ideals in $A$, i.e.
\[
  \spec(A) = \{\fp\subset A\colon \fp\text{ is prime}\} .
\]
The set $\spec(A)$ has a topology with closed sets 
\[
  V_\fa = \{\fp\in \spec(A)\colon \fp\supset \fa\} ,
\]
where $\fa$ ranges over ideals in $A$. An \emph{algebraic character} on $A$ is 
a ring homomorphism from $A$ to a field, $\chi\colon A\to K$. 

Given $\fp\in \spec(A)$, we get a character 
\[
  \chi_\fp\colon A\twoheadrightarrow A/\fp \hookrightarrow (A/\fp)_{(0)} .
\]
Conversely, given a character $\chi\colon A\to K$, the ideal 
$\ker(\chi)\in \spec(A)$. Two characters $\chi',\chi''$ give the same point in 
$\spec(A)$ if and only if a diagram \eqref{eq:equiv-characters} exists in the 
algebraic category.

There is a natural map $\ker\colon \Berkspec(A)\to \spec(A)$ (the \emph{kernel 
map}), given by $|\cdot|\mapsto \ker(|\cdot|)$. This should not be mistaken for 
the specialization map we will encounter later. 

\begin{lemma}
\leavevmode
\begin{enumerate}
\item
The map $\ker\colon \Berkspec(A)\to \spec(A)$ is continuous. 

\item
If $\|\cdot\|$ (the fixed norm on $A$) is trivial, then 
\begin{enumerate}
\item
$\ker\colon \Berkspec(A)\to \spec(A)$ is surjective, 

\item
there is a canonical section $\spec(A)\to \Berkspec(A)$
\end{enumerate}
\end{enumerate}
\end{lemma}
\begin{proof}
1. Let $\fa\subset A$ be an ideal. We claim that  
\[
  \ker^{-1}(V_\fa) = \bigcap_{f\in \fa}\{x\in \Berkspec(A)\colon |f|_x = 0\} .
\]
Indeed, if $x$ lies in the right hand side, then for all $f\in \fa$, 
$|f|_x=0$, whence $f\in \fa\Rightarrow f\in \fp_x$, so we get 
$\fa\subset \fp_x$, i.e.~$\fp_x\in V_\fa$. This is equivalent to 
$\ker(|\cdot|_x)\subset V_\fa$, i.e.~$x\in \ker^{-1}(V_\fa)$. In fact, all the 
above implications are equivalences.

2. Givn $\fp\in \spec(A)$, define $|\cdot|_\fp\in \Berkspec(A)$ by 
\[
  |f|_\fp = \begin{cases} 0 & f\in \fp \\ 1 & f\notin \fp \end{cases} .
\]
Since $\|\cdot\|$ is trivial, $|\cdot|_\fp$ is a bounded multiplicative 
norm on $A$. The map $\fp\mapsto |\cdot|_\fp$ is the desired canonical 
section. 
\end{proof}


\subsection{Functoriality of \texorpdfstring{$\Berkspec$}{M}}

We want to show that $A\mapsto \Berkspec(A)$ is functorial in the appropriate 
sense. Let $\varphi\colon (A,\|\cdot\|)\to (B,\|\cdot\|')$ be a bounded 
homomorphism of commutative Banach rings. (As always, we assume that 
$\varphi(1)=1$.)

\begin{theorem}
\leavevmode
\begin{enumerate}
\item
The map $\varphi^\ast\colon \Berkspec(B)\to \Berkspec(A)$ given by 
$\varphi^\ast(|\cdot|') = |\cdot|$, $|f| = |\varphi(f)|'$, is well-defined and 
continuous. 

\item
Assume the set 
\[
  \left\{\frac{\varphi(f)}{\varphi(g)}\colon f,g\in A\text{ and }\varphi(g)\in B^\times\right\} 
\]
is dense in $B$. Then $\varphi^\ast$ is injective. 

\item
If $\varphi^\ast$ is injective, then $\Berkspec(B)$ is homeomorphic to its 
image in $\Berkspec(A)$. 
\end{enumerate}
\end{theorem}
\begin{proof}
1. That $|\cdot| = \varphi^\ast(|\cdot|')$ is multiplicative is obvious. To see 
that it is bounded, note that for all $f\in A$, we have 
\[
  |f| = |\varphi(f)|' \leqslant c_1 \|\varphi(f)\|' \leqslant c_1 c_2 \|f\| .
\]
Continuity follows from the fact that 
\[
  (\varphi^\ast)^{-1}\left(\{x\in \Berkspec(A)\colon |f(x)|>\alpha\}\right) = \{y\in \Berkspec(B)\colon |\varphi(f)(y)| > \alpha\} .
\]

2. Let $|\cdot|'\ne|\cdot|''\in \Berkspec(B)$ be such that 
$\varphi^\ast(|\cdot|') = \varphi^\ast(|\cdot|'')$. Then there exists $h\in B$ 
such that $|h|'\ne |h|''$, so we may as well assume $|h|'<|h|''$. Let 
$2\epsilon < |h|''-|h|'$. By assumption, there exists $f,g\in A$ such that 
$\varphi(g)\in B^\times$ and 
$\left\|h-\frac{\varphi(f)}{\varphi(g)}\right\|' < \epsilon$. Thus 
$\left|h-\frac{\varphi(f)}{\varphi(g)}\right| < \epsilon$ and 
$\left|h-\frac{\varphi(f)}{\varphi(g)}\right|'' < \epsilon$. Apply the triangle 
inequality:
\begin{align*}
  \left|\frac{\varphi(f)}{\varphi(g)}\right|' 
    &= \left|\frac{\varphi(f)}{\varphi(g)} - h+ h\right|' \\
    &\leqslant \left|\frac{\varphi(f)}{\varphi(g)} - h\right|' + |h|' \\
    &< \epsilon + |h|' \\
    &< |h|' - \epsilon .
\end{align*}
Similarly 
\begin{align*}
  |h|'' 
    &= \left|h - \frac{\varphi(f)}{\varphi(g)} + \frac{\varphi(f)}{\varphi(g)}\right|'' \\
    &\leqslant \left|h-\frac{\varphi(f)}{\varphi(g)}\right|'' + \left|\frac{\varphi(f)}{\varphi(g)}\right|'' \\
    &< \epsilon + \left|\frac{\varphi(f)}{\varphi(g)}\right|'' .
\end{align*}
It follows that 
$\left|\frac{\varphi(f)}{\varphi(g)}\right|' < \left|\frac{\varphi(f)}{\varphi(g)}\right|''$, 
so $|\varphi(f)|' |\varphi(g)|'' < |\varphi(f)|'' |\varphi(g)|'$. This implies 
$|f||g| < |f| |g|$, a contradiction. 

3. This follows from the general fact, proved in [Folland, p.129], that if 
$X$ is quasi-compact, $Y$ is Hausdorff, then any continuous bijection 
$f\colon X\to Y$ is a homeomorphism. 
\end{proof}

Note that the recurring norm 
\[
  |f|_\fp = \begin{cases} 1 & f\notin \fp \\ 0 & f\in \fp \end{cases} ,
\]
comes from the trivial seminorm on $A/\fp$. 



\subsection{Boundedness versus continuity for \texorpdfstring{$K$}{K}-algebras}

Usually (e.g., in classical algebraic geometry) one cares about rings that are 
also $K$-vector spaces, for some field $K$. 

\begin{definition}
Let $R$ be a commutative ring. A (unital, associative) \emph{$R$-algebra} is an 
abelian group $A$ equipped with the structure of both a (associative, unital) 
ring and an $R$-module, such that ring multiplication is $R$-bilinear in the 
sense that $r\cdot(x y) = (r\cdot x) y = x(r\cdot y)$. 
\end{definition}

It is equivalent to specify a (unital, associative) ring $A$ together with a 
ring homomorphism $R\to \zentrum(A)$, where 
$\zentrum(A)=\{a\in A\colon a b=ba\text{ for all }b\in A\}$ is the center of 
$A$. We are mainly interested in commutative $R$-algebras, but some aspects of 
the theory (especially over $\bC$) work just as well for possibly 
non-commutative algebras. 

Let $K$ be a field with absolute value, $A$ a $K$-algebra. A \emph{norm} 
$|\cdot|$ on $A$ is both a ring norm and a $K$-vector space norm. This in 
particular requires that $\|a v\| = |a| \| v\|$ for all $a\in K$, $v\in A$. 
So any norm on $A$ ``remembers'' the norm on $K$. More precisely, if $A$ is a 
normed $K$-algebra, $\|\cdot\|_x\colon A\to \bR$ a bounded multiplicative 
seminorm. Then $|\cdot|_x$ induces an absolute value on $\sH(x)$. But $\sH(x)$ 
is a complete extension of $K$. It turns out that $|\cdot|_x$ on $\sH(x)$ 
must restrict to $|\cdot|$ on $K$. This is Exercise 4.3.1 in Conrad's notes in 
\cite{aws-2008}.

Let $\varphi\colon A\to B$ be a $K$-algebra homomorphism. We assume that 
$A$ and $B$ have been given ($K$-)norms $\|\cdot\|$ and $\|\cdot\|'$. Clearly, 
if $\varphi$ is bounded (i.e.~$\|\varphi(f)\|' \leqslant c\|f\|$ for all 
$f\in A$), then $\varphi$ is continuous. We have seen that the converse does 
not always hold. 

Let's try working towards proving the converse. Let $\varphi\colon A\to B$ be a 
continuous homomorphism. Then $\varphi$ is continuous if and only if it is 
continuous at $0$. So there exists a neighborhood $U\ni 0$ such that 
$\varphi(U)\subset \{b\in B\colon \|b\|'<1\}$. There is $\delta>0$ such that 
$\{a\in A\colon \|a\|\leqslant \delta\}\subset U$. So 
$\|\varphi(x)\|'\leqslant 1$ whenever $\|x\|\leqslant \delta$. If we were 
working over $\bC$, we would not rescale. But this doesn't work if the norm on 
$K$ is trivial. If the norm $|\cdot|$ on $K$ is \emph{not} trivial (i.e., 
$|K^\times|\ne 1$) then by rescaling we can say that 
$\|x\|\leqslant |a|\Rightarrow \|\varphi(x)\|' \leqslant \frac{|a|}{\delta}$. 
So generally, $\|\varphi(x)\|' \leqslant \frac{1}{\delta}\|x\|$, and 
$\varphi$ is bounded. 

If $|\cdot|$ is trivial on $K$, then continuity does not imply boundedness. If 
$|\cdot|$ is nontrivial, many other (but not all) basic facts of functional 
analysis are also true, e.g.~Banach's open mapping theorem (a bijective bounded 
$K$-algebra homomorphism $A\to B$ has bounded inverse). 





\subsection{Gelfand's theory for \texorpdfstring{$K=\bC$}{K=C}}

Much of this material can be found in \cite[Ch.~10-11]{rudin-1991}. 
Here is our motivating example. 

\begin{example}
Let $X$ be a nonempty compact Hausdorff space. Let $A=\sC^0(X)$ be the space of 
continuous $\bC$-valued functions on $X$. With pointwise multiplication, $A$ is 
naturally a (commutative, unital) $\bC$-algebra. We give $A$ the supremum norm: 
\[
  \|f\| = \sup_{x\in X} |f(x)| .
\]
It's a standard fact that $A$ is complete, i.e.~it is a Banach $\bC$-algebra. 
\emph{Warning}: the norm on $\sC^0(X)$ is \emph{not} in general multiplicative, 
i.e.~we only have $\|fg\|\leqslant \|f\|\|g\|$.
\end{example}

\begin{example}
Let $X$ be a finite set with $n$ points. Then $C^0(X)=\bC^n$. 
\end{example}

It is natural to ask: ``given any commutative Banach $\bC$-algebra $A$, can we 
embed $A$ into $\sC^0(X)$ for some compact Hausdorff $X$?''

\begin{example}
We give $\Lp^1(\bR^n)$ the structure of a $\bC$-algebra via convolution:
\[
  (f\ast g)(x) = \int_{\bR^n} f(y) g(x-y)\, \mathrm{d} y .
\]
This makes $\Lp^1(\bR^n)$ a commutative, but non-unital algebra. We give 
$\Lp^1(\bR^n)$ a unit as follows. Let $A=\Lp^1(\bR^n)\oplus\bC\delta$, where 
$\delta$ is the ``Dirac delta'' with multiplication as follows:
\[
  (f_1+\lambda_1\delta) \ast (f_2+\lambda_2 \delta) = (f_1\ast f_2+\lambda_2 f_1 + \lambda_1 f_2) + \lambda_1 \lambda_2 \delta .
\]
Give $A$ a norm by 
\[
  \|f+\lambda\delta\| = \|f\|_{\Lp^1} + |\lambda| .
\]
\end{example}

\begin{example}
Let $X$ be a Banach space over $\bC$. Let $\sB(X)$ be the algebra of bounded 
linear operators on $X$, with $\|\cdot\|$ the operator norm. For example, if 
$\dim(X)=n$, then $\sB(X)\simeq M_n(\bC)$. Note that if $n>1$, then $\sB(X)$ is 
\emph{not} commutative. The theory of Gelfand-Mazur (and some of Berkovich's) 
work for non-commutative algebras as well. 
\end{example}

\begin{definition}
Let $A$ be a $\bC$-algebra. Put 
\[
  \sigma(f) = \{\lambda\in \bC\colon \lambda-f\notin A^\times\} .
\]
(In functional analysis, one calls $\sigma(f)$ the \emph{spectrum} of $f$, but 
we will avoid using this terminology. The complement 
$\bC\smallsetminus \sigma(f)$ is called the \emph{resolvent} of $f$.)
\end{definition}

For $K$ non-archimedean, one can give a definition for $\sigma(f)$, but it's 
quite a bit more complicated, so we won't reproduce it here. See 
\cite[Ch.~7]{berkovich-1990} for details. 

\begin{theorem}
Let $f\in A$. Then 
\begin{enumerate}
\item
$\sigma(f)\ne\varnothing$, 

\item
$\sigma(f)$ is compact. 
\end{enumerate}
\end{theorem}
\begin{proof}
Part 1 is essentially \autoref{lemma:spec-Banach}. For part 2, it suffices to 
show that $\sigma(f)$ is closed and bounded. That $\sigma(f)$ is closed follows 
(essentially) from the fact that $A^\times$ is open. For boundedness of 
$\sigma(f)$, see below. 
\end{proof}

\begin{definition}
For $f\in A$, put 
\[
  \rho(f) = \sup\{|\lambda|\colon \lambda\in \sigma(f)\} .
\]
\end{definition}

\begin{theorem}
$\rho(f) = \lim_{n\to \infty} \|f^n\|^{1/n}$. 
\end{theorem}
The proof will be given in a more general setting, after giving an equivalent 
definition of $\rho(f)$. 

\begin{definition}
Let $A$ be a Banach $\bC$-algebra. As a set, 
\[
  \cM(A) = \{\chi\colon A\to \bC\text{ a homomorphism of $\bC$-algebras}\} .
\]
\end{definition}

\begin{lemma}
\leavevmode
\begin{enumerate}
\item
Any $\chi\in \cM(A)$ is bounded.

\item
$\cM(A)$ can be identified with the space of maximal ideals in $A$ via 
$\chi\leftrightarrow \ker(\chi)$. 

\item
$\cM(A)=\Berkspec(A)$

\item
$f\in A^\times$ if and only if $\chi(f)\ne 0$ for all $\chi\in \cM(A)$. 

\item
$\lambda\in \sigma(f)$ if and only if $\chi(f)=\lambda$ for some 
$\chi\in \cM(A)$. 
\end{enumerate}
\end{lemma}
\begin{proof}
1. Use the fact that $\chi\colon A\to \bC$ respects the $\bC$-algebra structure 
of $A$. 

2. For $x\in \cM(A)$, $\sH(x)$ is a complete field containing $\bC$, we have 
$\sH(x)=\bC$. 

4. If $f\notin A^\times$, then $f\notin \fm$ for all $\fm$ maximal. 

5. Apply part 4 to $\lambda-f$. 
\end{proof}

Parts 2 and 3 essentially say that ``Berkovich analytification'' does 
\emph{not} produce any new points when we endow $\bC$ with the archimedean 
absolute value. When we endow $\bC$ with the trivial absolute value, the 
Berkovich analytification carries a lot of extra information. 

For $f\in A$, the \emph{Gelfand transform} of $f$ is the function 
$\hat f\colon \cM(A)\to \bC$ defined by 
\[
  \hat f(\chi) = \chi(f) .
\]
The Gelfand transform is \emph{a priori} a function $A\to \prod_{\cM(A)}\bC$. 
We have seen that $\rho(f) = \sup_{\chi\in \cM(A)} |\hat f(\chi)|$. Give 
$\cM(A)$ the weakest topology making each $\hat f$ continuous (this recovers 
the topology on $\Berkspec(A)$ we defined earlier). Then the Gelfand 
transform is a map $A\to \sC^0(\cM(A))$ and 
\[
  \rho(f) = \|\hat f\|_{\sC^0(\cM(A))} .
\]
Let $B=\{\hat f\colon f\in A\}\subset \sC^0(\cM(A))$. Clearly, the Gelfand 
transform gives us a $\bC$-algebra homomorphism 
$A\twoheadrightarrow B\hookrightarrow \sC^0(\cM(A)),f\mapsto \hat f$. What is 
the kernel of $A\twoheadrightarrow B$? Suppose $\hat f=0$, 
i.e.~$\hat f(\chi)=0$ for all $\chi$, i.e.~$\chi(f)=f\mod\ker(\chi)=0$ for all 
$\chi$. In other words, $\hat f=0$ if and only if $f\in \fm$ for all maximal 
ideals $\fm$ of $A$. So 
\[
  \ker(\hat\cdot) = \bigcap_{\fm\in \mathrm{mSpec}(A)} \fm = \text{Jacobson radical of } A  = \ker(\rho).
\]
So $\hat\cdot$ is injective if and only if $\rho$ is a norm (not just a 
seminorm) on $A$, if and only if $A$ is Jacobson semisimple. (We say a algebra 
$A$ is \emph{Jacobson semisimple} if its Jacobson radical vanishes.)

The algebra $A$ has a norm $\|\cdot\|$ already, while 
$B\subset \sC^0(\cM(A))$ carries the sup-norm. The map $\hat\cdot\colon A\to B$ 
is an sometry if and only if $\rho(\cdot)=\|\cdot\|$. Since $\rho$ is 
power-multiplicative, clearly a necessary condition is for $\|\cdot\|$ to be 
power-multiplicative. We will see later that $\rho(\cdot)=\|\cdot\|$ if and 
only if $\|\cdot\|$ is power-multiplicative. 

\begin{definition}[Old]
A $\bC$-Banach algebra $A$ is called \emph{uniform} if there exists a compact 
Hausdorff space $X$ such that there is a bounded $\bC$-algebra homomorphism
$A\hookrightarrow \sC^0(X)$ with image a closed subspace containing the 
constant functions and separating points of $X$. 
\end{definition}

(Recall that $S\subset \sC^0(X)$ \emph{separates points} if for any 
$x\ne y\in X$, there is $s\in S$ such that $s(x)\ne s(y)$.) An algebra 
$(A,\|\cdot\|)$ is uniform if and only if $\rho(\cdot)=\|\cdot\|$, if and only 
if $\|\cdot\|$ is power-multiplicative. 

Given a Banach $\bC$-algebra $(A,\|\cdot\|)$, one can always find a related 
uniform algebra $(A^\mathrm{u},\rho(\cdot))$. Namely, $A^\mathrm{u}$ is the 
completion of $A/\ker(\rho)$ with respect to the norm $\rho$. 

\begin{example}
Let $A=\Lp^1(\bR^n)\oplus \bC\delta$. Given any $\omega\in \bR^n$, we have a 
character $\chi_\omega\colon A\to \bC$, defined by 
\[
  \chi_\omega(f+\lambda\cdot\delta) = \hat f(\omega) + \lambda .
\]
Here, $\hat f$ is the usual Fourier transform of $f$. Also, there is the 
character 
\[
  \chi_\infty(f+\lambda\cdot \delta) = \lambda .
\]
It is known that 
$\cM(A) = \{\chi_\omega\}_{\omega\in \bR^n}\cup\{\chi_\infty\}$. So 
$\cM(A)$ is the ``spectrum'' space. Moreover, the (weak) topology on $\cM(A)$ 
is the same as the one-point compactification of $\bR^n$. 
\end{example}

As an exercise, make $[0,1]$ out of $[0,1]\cap \bQ$ by announcing what you 
would like to be your continuous functions on $[0,1]$. For a more formal 
discussion, see the introduction to \cite{berkovich-1990}. 

The theory of Gelfand-Mazur has another analytic generalization to 
C*-algebras and non-commutative geometry in the sense of A.~Connes. The 
``spectrum'' can be replaced by the \emph{unitary dual}, which plays a large 
role in the representation theory of real reductive groups. 





\subsection{Banach rings: the general case}

Let $A$ be a Banach ring, $f\in A$. 

\begin{definition}\label{def:general-spectral-radius}
The \emph{spectral radius} of $f$ is 
\[
  \rho(f) = \lim_{n\to \infty} \|f^n\|^{1/n} = \inf\{\|f^n\|^{1/n}\colon n\geqslant 1\} .
\]
\end{definition}

\begin{lemma}
\autoref{def:general-spectral-radius} makes sense. 
\end{lemma}
\begin{proof}
This follows from \autoref{thm:fekete} applied to $a_n=\log\|f^n\|$. The 
sequence is subadditive because $\|f^{n+m}\| \leqslant \|f^n\| \|f^m\|$.
\end{proof}

\begin{theorem}[Fekete]\label{thm:fekete}
Let $\{a_n\}_{n\geqslant 1}\subset \bR\cup\{-\infty\}$. If $\{a_n\}$ is 
subadditive ($a_{n+m}\leqslant a_n+a_m$), then 
$\lim_{n\to \infty} \frac{a_n}{n}$ exists and is equal to 
$\inf\{\frac{a_n}{n}\colon n\geqslant 1\}$. 
\end{theorem}
\begin{proof}
If $\inf\{\frac{a_n}{n}\colon n\geqslant 1\}=-\infty$, then it easily follows 
that $\lim \frac{a_n}{n}=-\infty$. 

Suppose the infimum is finite, and let 
$a=\inf\{\frac{a_n}{n}\colon n\geqslant 1\}$. Fix $\epsilon>0$, and let 
$k\gg 0$ be such that $\left|\frac{a_k}{k}-a\right|<\epsilon/2$. Let $l\gg 0$ 
be such that $\frac{a_r}{kl}<\epsilon/2$ for $r<k$. If $n>k l$, write  
$n=kq+r$ with $r<k$. Then $q\geqslant l$, so 
\[
  a \leqslant \frac{a_n}{n} \leqslant \frac{a_{kq}+a_r}{kq+r} \leqslant \frac{q a_k + kl\epsilon/2}{kq} = \frac{a_k}{k} + \frac{l\epsilon}{2q} < a+\epsilon .
\]
\end{proof}

Recall the Gelfand transform 
$\hat\cdot\colon A\twoheadrightarrow B\subset \prod^\mathrm{b}_{x\in \Berkspec(A)} \sH(x)$, given by 
\[
  \hat f(x) = f\mod{\ker|\cdot|_x} .
\]
The algebra $B$ inherits the supremum norm.  

\begin{theorem}[Spectral radius vs.~Berkovich spectrum]
Let $A$ be a commutative Banach ring. Then for all $f\in A$, 
$\rho(f) = \|\hat f\|$, i.e.~
\[
  \lim_{n\to \infty} \|f^n\|^{1/n} = \max_{x\in \Berkspec(A)} |f(x)| .
\]
\end{theorem}
\begin{proof}
First we show that 
$\lim_{n\to \infty} \|f^n\|^{1/n} \geqslant \max_{x\in \Berkspec(A)} |f(x)|$. 
Let $f\in A$ and $x\in\Berkspec(A)$. Then
\[
  \|f^n\| \geqslant |f^n|_x=|f|_x^n=|f(x)|^n,
\]
so $\rho(f)\geqslant |f(x)|$ and therefore $\rho(f)\geqslant \|\hat f\|$.

To show $\lim_{n\to \infty} \|f^n\|^{1/n} \leqslant \max_{x\in \Berkspec(A)} |f(x)|$, 
we will prove that for any $r\in \bR^{\geqslant 0}$, $\|\hat f \|<r$ implies 
$\rho(f)<r$. Fix $r$, and assume $\|\hat f\|<r$. Then, by boundedness, 
$|f(x)|<r$ for all $x\in\Berkspec(A)$. Let
\[
  A^\prime=\left\{ \sum_{i=0}^\infty a_iT^i : a_i\in A, \sum_{i=0}^\infty a_ir^{-i}<\infty \right\}
\]
be the ring of convergent power series with coefficients in $A$ having radius 
of convergence $r$. Note that $\sum_{i=0}^\infty a_ir^{-i}$ is a norm on 
$A^\prime$.

We claim that $1-fT$ is invertible in $A^\prime$. Assuming this claim, we have
\begin{align*}
  1-fT\text{ is invertible } 
    &\Leftrightarrow \sum_{i=0}^\infty f^iT^i\in A^\prime \\
    &\Leftrightarrow \sum_{i=0}^\infty a_ir^{-i}<\infty \\
    &\Rightarrow \| f^i \|r^{-i}<1 \text{ for }i\text{ sufficiently large} \\
    &\Rightarrow \|f^i\|^{\frac{1}{i}}<r \\
    &\Rightarrow \rho(f)<r,
\end{align*}
and we are done. To prove the claim, note that 
\begin{align*}
  1-fT\text{ is invertible in }A^\prime 
    &\Leftrightarrow (1-fT)(x)\neq 0 \;\forall x\in M(A^\prime) \\
    &\Leftrightarrow |1-fT|_x\neq 0 \;\forall x\in M(A^\prime)
\end{align*}
To show this it suffices to show $|fT|_x<1$ because
\[
  1=|1|_x=|1-fT+fT|_x\leqslant |1-fT|_x+|fT|_x
\]
and if $|1-fT|_x=0$ then the above inequality implies $1<0+1$.

To compute $|fT|_x$, note that $\|T\|=r^{-1}$. By assumption, we have $|f|_x<r$ 
for all $x\in \Berkspec(A)$. But the map
\[
  \Phi:A\to A^\prime
\]
sending $f\mapsto f$ induces an isometry
\[
  \Phi^\ast:\Berkspec(A^\prime)\to\Berkspec(A),
\]
so $|f|_x=|f|_{\Phi^*(x)}<r$ for all $x\in \Berkspec(A^\prime)$, and therefore
\[
|fT|_x=|f|_x|T|_x<rr^{-1}=1.
\]
\end{proof}
The following theorem shows that $\rho(\cdot)$ is a canonical seminorm
\begin{theorem}\label{thm:rho}
Let $(A,\|\cdot\|)$ be a commutative Banach algebra. Then
\begin{enumerate}
\item $\rho$ only depends on the equivalence class of $\|\cdot \|$.
\item $\rho:A\to\bR$ is a seminorm.
\item $\rho$ is always power-multiplicative.
\item $\rho(f)\leqslant\|f\|\;\forall f\in A$, moreover, $\rho(\cdot)=\|\cdot\|$ 
if and only if $\|\cdot\|$ is power-multiplicative.
\end{enumerate}
\end{theorem}
These results can be restated in terms of properties of the Gelfand transform
\begin{theorem}\label{thm:Gelfandtransform}
Let $\hat \cdot:A\to B$ be the Gelfand transform. Then
\begin{enumerate}
\item[(a)] $\ker \hat\cdot =\ker \rho$.
\item[(b)] $\hat\cdot$ is an isometry with respect to $(A,\|\cdot\|)$ and 
$(B,\rho(\cdot)$ if and only if $\|\cdot\|$ is power-multiplicative.
\end{enumerate}
\end{theorem}
The ideal $\ker\rho$ is called the \emph{quasinilradical} of $A$. Its elements 
are the quasinilpotents of $A$, i.e.~those elements $f$ of $A$ for which
\[
  f(x)=0 \;\forall x\in \Berkspec(A).
\]
In algebraic geometry, $f\in R$ is identically zero on $\spec(R)$ if and only 
if
\[
  f(\fp)=f+\fp=0\quad \forall\fp\in\spec(R).
\]
which happens if and only if
\[
  f\in\bigcap_{\fp\in\spec(R)}\fp=\text{rad}(\langle 0\rangle),
\]
which is known as the nilradical of $R$.

Note that the nilradical is contained in the quasinilradical because of the 
power-multiplicativity of $\rho$. Recall that we have shown that for 
$\bC$-Banach algebras, the quasinilradical is equal to the Jacobson radical of 
$A$.

\begin{definition}\label{def:uniform}
A Banach algebra $(A,\|\cdot\|)$ is called \emph{uniform} if $\|\cdot\|$ is 
power-multiplicative.
\end{definition}
\begin{lemma}\label{lem:squaresonly}
$\|\cdot \|$ is power-multiplicative if and only if for all $f\in A$, 
$\| f^2\|=\|f\|^2$.
\end{lemma}
\begin{proof}
Fix $f$. Define $\Phi(n)=\log \|f^n\|$. By assumption, we have
\[
  \Phi(2)=2\Phi(1)
\]
so $\Phi(2^m)=2^m\Phi(1)$. We want $\Phi(n)=n\Phi(1)$. Note: $\Phi$ is 
subadditive, so $\Phi(n)\leqslant n\Phi(1)$. If $\Phi(n)<n\Phi(1)$, then let 
$k$ be such that $k+n=2^m$. Then
\[
  2^m\Phi(1)=\Phi(2^m)=\Phi(k+n)\leqslant \Phi(k)+\Phi(n)<k\Phi(1)+n\Phi(1)=2^m\Phi(1)
\]
which is a contradiction.
\end{proof}

\begin{proof}[Proof of \autoref{thm:rho}]
Let 
\begin{align}\label{eq:def1}
  \rho &= \lim_{n\to \infty} \|f^n\|^{1/n}) \\ \label{eq:def2}
  \rho &= \max_{x\in \Berkspec(A)} |f(x)|
\end{align}
denote the two equivalent definitions of $\rho$. 
\begin{enumerate}
\item
This follows immediately from \eqref{eq:def1}.

\item
$\rho(0)=0,\rho(1)=1$ and $\rho(fg)\leqslant\rho(f)\rho(g)$ follow from 
\eqref{eq:def1}, $\rho(f+g)\leqslant\rho(f)+\rho(g)$ follows easily from 
\eqref{eq:def2}.

\item
\eqref{eq:def1}

\item
By \eqref{eq:def1}, we have 
$\rho(f)=\lim_{n\to\infty} \|f^n\|^{\frac{1}{n}}\leq \lim_{n\to\infty}\|f\|^{n\frac{1}{n}}=\|f\|$. 
If $\rho=\|\cdot\|$, then by 3, $\|\cdot\|$ is power-multiplicative, so 
$\rho(f)=\|f^n\|^{\frac{1}{n}}=\|f\|$.
\end{enumerate}
\end{proof}
\begin{proof}[Proof of \autoref{thm:Gelfandtransform}]
\begin{enumerate}
\item[(a)] Follows from \eqref{eq:def1}.
\item[(b)] Follows from \autoref{thm:rho}, 4.
\end{enumerate}
\end{proof}
Recall that the kernel map $\ker\colon\Berkspec(A)\to\spec(A)$ sending 
$x\mapsto \ker |\cdot|_x$ is continuous with respect to the Berkovich topology 
on $\Berkspec(A)$ and the Zariski topology on $\spec(A)$.

\begin{definition}\label{def:zariskiberkspec}
The Zariski topology on $\Berkspec(A)$ is the weakest topology making $\ker$ 
continuous.
\end{definition}

As an exercise, show that $\Berkspec(A)$ is irreducible with respect to the 
Zariski topology if and only if $\ker\rho$ is a prime ideal.


\subsection{Uniformization of Banach Rings}

Let $(A,\|\cdot\|)$ be a Banach Ring. We would like to find a uniform 
Banach Ring associated to $A$. Motivated by \autoref{thm:rho}, we would like to 
use $\rho$, but as seen in that theorem, $\rho$ is in general only a seminorm, 
and $A/\ker\rho$ might not be complete. Therefore we define the uniformization 
$A^\mathrm{u}$ of $A$ to be
\[
  A^\mathrm{u}=\widehat{A/\ker \rho}
\]
where $\hat{\cdot}$ refers to completion with respect to $\rho$. Then 
$(A^\mathrm{u},\rho)$ is a uniform Banach ring (we denote the residue norm of 
$\rho$ by $\rho$ again). Let $\pi\colon A\to A^\mathrm{u}$ be the quotient map.

\begin{theorem}\label{thm:uniformization}
\leavevmode
\begin{enumerate}
\item For any bounded morphism $\phi\colon A\to B$ with $B$ uniform we have
\[
\begin{tikzcd}
  A \ar[dr, "\phi"] \ar[r, "\pi"]
    & A^\mathrm{u} \ar[d, dotted, "\exists !"] \\
  & B
\end{tikzcd}
\]
\item We have $\Berkspec(A^\mathrm{u}) = \Berkspec(A)$.
\item $\Berkspec(A^\mathrm{u})$ has one point if and only if $A^\mathrm{u}$ is 
a field and $\rho$ is an absolute value on $A^\mathrm{u}$.
\item $\rho$ is non-archimedean on $A^\mathrm{u}$ if and only if for all 
$x\in \Berkspec(A^u)$, $(\sH(x),{|\cdot|_x})$ is non-archimedean.
\end{enumerate}
\end{theorem}
\begin{proof}
1. By a similar trick as before, using power-multiplicativity of $\|\cdot\|_B$, we may assume
\[
  \|\phi(f)\|_B\leqslant \rho(f).
\]
So if $\rho(f)=0$ then $\phi(f)=0$ and $\phi\colon A\to B$ factors through 
$\pi\colon A\to A^\mathrm{u}$.

2. The map $\pi\colon A\to A^\mathrm{u}$ is bounded since it's a quotient map. Then
\[
  \pi^\ast\colon \Berkspec(A^\mathrm{u})\to\Berkspec(A)
\]
is injective, since $\pi(A)$ is dense in $A^\mathrm{u}$. Both $\Berkspec(A^\mathrm{u})$ 
and $\Berkspec(A)$ are compact Hausdorff spaces, so injectivity of $\pi$ implies
\[
  \Berkspec(A^\mathrm{u})\simeq \pi\left( \Berkspec(A^\mathrm{u}) \right),
\]
so we just need to show that $\pi^\ast$ is surjective. Let 
$|\cdot|_x\in \Berkspec(A)$. Note that it suffices to give such a 
$|\cdot|_y\in\Berkspec(A/\ker\rho)$, since then $|\cdot|_y$ will extend uniquely 
to the completion $A^\mathrm{u}$. Define
\[
  |f+\ker\rho|_y=|f|_x=|f(x)|.
\]
This is well-defined as $\rho(f)=\max_{x\in\Berkspec(A)}|f(x)|=0$ implies $f(x)=0$ 
for all $x$.

4. Assume that for all $x\in\Berkspec(A^\mathrm{u})$, $(\sH(x),|\cdot|_x)$ is 
non-archimedean. Then as
\[
  \rho(f)=\max_{x\in\Berkspec(A^\mathrm{u})}|f(x)|=\max_{x\in\Berkspec(A^\mathrm{u})}|f|_x,
\]
$\rho$ is non-archimedean.

For the converse, recall that it suffices to check the archimedean property for 
integers. Assume that $\rho$ is non-archimedean. Then $\rho(n)\leq 1$ for all 
$n\in\bZ$. Therefore for $x\in\Berkspec(A^\mathrm{u})$,
\[
  |n|_x\leqslant\rho(n)\leqslant 1
\]
so $|\cdot|_x$ is non-archimedean.
\end{proof}



\subsection{Products}

Let $(A,\|\cdot\|_A)$ be a normed ring.
\begin{definition}\label{def:modules}
A \emph{seminorm} on an  $A$-module $M$ is a function
\[
  \|\cdot\|:M\to\bR
\]
such that
\begin{enumerate}
\item
\begin{itemize}
\item $\|0\|=0,$
\item $\|m+n\|\leqslant \|m\|+\|n\|,$
\item $\|m\|=\|-m\|.$
\end{itemize}
\item There exists some $c>0$ such that for all $f\in A,m\in M$,
\[
\|f\cdot\|\leqslant c\|f\|_A\|m\|.
\]
\end{enumerate}
\end{definition}
\emph{Fact:} We may assume $c=1$ by replacing $\|\cdot\|$ by $\|\cdot\|^\prime$, defined as
\[
\|m\|^\prime=\sup_{f\in A\setminus\{0\}}\frac{\|f\cdot m\|}{\|f\|_A}.
\]
Then
\[
\|m\|\leqslant \|m\|^\prime\leqslant c\|m\|
\]
so $\|\cdot\|$ is equivalent to $\|\cdot\|^\prime$.

Let $M$ and $N$ be $A$-modules. On the tensor product $M\otimes_A N$ we have some natural choices for seminorms.
\begin{enumerate}
\item If $A,M,N$ are all archimedean, then for $s\in M\otimes_A N$, let
\begin{align}
\|s\|&=\inf \left( \sum_{i\in I}\|m_i\|\|n_i\| \right),\label{def:archtensor}
\end{align}
where we are taking the infimum of the above for all presentations of $s$:
\begin{align}
s&=\sum_{i\in I} m_i\otimes n_i.\label{def:nonarchtensor}
\end{align}
\item If $A,M,N$ are all non-archimedean, we prefer the following choice
\[
\|s\|=\inf \left( \text{Max}_{i\in I} \|m_i\|\|n_i\| \right)
\]
\end{enumerate}
Some remarks about this definition:
\begin{enumerate}
\item Definition \autoref{def:archtensor} also makes sense in the non-archimedean setting, but the two definitons are nonequivalent in general.
\item Definition \autoref{def:nonarchtensor} will not work in the archimedean case.
\item Even when the seminorms on $M$ and $N$ are norms, the seminorm on $M\otimes_A N$ might have a nontrivial kernel.
\item In general $M\otimes_A N$ is not complete.
\end{enumerate}
To remedy this, we define
\[
M\hat{\otimes}_AN=\widehat{M\otimes_A N}
\]
where completion is taken with respect to the above-defined norm. Then $M\hat{\otimes}_AN$ is also an $\hat{A}$-module.

Note that the map
\[
M\otimes_A N\to M\hat{\otimes}_AN
\]
is not in general injective.
\begin{theorem}[Gruson] Let $K$ be a non-archimedean complete field, $M$ and $N$ be $K$-Banach vector spaces. then
\[
M\otimes_A N\to M\hat{\otimes}_AN
\]
is injective.
\end{theorem}
The completed tensor product $M\hat{\otimes}_AN$ also has the following universal property: Any bounded bilinear map $M\times N\to L$ where $L$ is a complete normed $A$-module factors through the canonical map
\[
M\times N\to M\hat{\otimes}_A N.
\]
Our main interest in tensor products is extension of scalars, i.e. if $A$ is a $K$-Banach algebra and $L/K$ is a field extension, we want to study
\[
A^\prime=A\hat{\otimes}_K L
\]
