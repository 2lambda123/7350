% !TEX root = 7350.tex

\section{Berkovich spaces}

This generalizes Gelfand's theory. Much of the theory even works for 
non-commutative algebras. We do not require the existence of a base field, so 
our theory applies to ``arithmetic'' rings like $\bZ$ or 
$\bZ_2[x]/(7 x^3-5)$. 





\subsection{Norms and seminorms}

We could phrase everything in terms of (rank-one) valuations and semivaluations.
Let $A$ be a commutative (unital) ring. 

\begin{definition}
A \emph{seminorm} on $A$ is a function $\|\cdot\|\colon A\to \bR^{\geqslant 0}$ 
satisfying 
\begin{enumerate}
\item
$\|0\|=0$, $\|1\|\leqslant 1$, 

\item
$\|f+g\|\leqslant \|f\|+\|g\|$, 

\item
$\|f g\| \leqslant \|f\| \|g\|$. 
\end{enumerate}
\end{definition}

If we replace axiom 2 with $\|f+g\| \leqslant \max\{\|f\|,\|g\|\}$, we say the 
norm $\|\cdot\|$ is non-archimedean. We will focus primarily on non-archimedean 
norms. We allow $x\ne 0$ to have $\|x\|=0$. Define 
\[
  \ker(\|\cdot\|) = \{x\in A\colon \|x\| = 0\} .
\]
(Huber calls this the \emph{support} of $\|\cdot\|$.) This is a prime ideal in 
$A$. We call $\|\cdot\|$ a \emph{norm} if $\ker(\|\cdot\|)=0$. It turns out 
that our hypotheses imply that either $\|1\|=0$ or $\|1\|=1$. For, if 
$\|1\|\ne 0$, then $\|1\| \leqslant \|1\|\|1\|$, so $1\leqslant \|1\|$. If 
$\|1\|=0$, then $\|f\|=0$ for all $f\in A$. We want to allow this possibility 
so that the zero ring can have a norm. 

If $(A,\|\cdot\|)$ is a ring with seminorm, then $(A,+)$ is naturally a 
seminormed group, i.e.~it satisfies 
\begin{align*}
  \|0\| &=0 \\
  \|f+g\| &\leqslant \|f\|+\|g\| \\
  \|-f\| &= \|f\| .
\end{align*}
(Alternatively, replace the last two equations by 
$\|f-g\| \leqslant \|f\|+\|g\|$.) Any seminormed group carries a natural 
topology, so we get a topology on $A$. This topology is Hausdorff if and only 
if $\ker(\|\cdot\|)=0$. The (Hausdorff) completion $\widehat A$ exists (it can 
be defined via Cauchy sequences as usual), but the obvious map 
$Ai\colon\to \widehat A$ is injective if and only if $\ker(\|\cdot\|)=0$. In 
fact, $|ker(i)=\ker(\|\cdot\|)$. 

If we replace the axiom $\|f g\| \leqslant \|f\| \|g\|$ with 
$\|f g\| = \|f\| \|g\|$ and also assume $\|\cdot\|\ne 0$, we say that 
$\|\cdot\|$ is \emph{multiplicative}. If for all $f\in A$, $\|f^n\| = \|f\|^n$, 
we say $\|\cdot\|$ is \emph{power-multiplicative}. 

If $N$ is a subgroup of a seminormed group $(A,+)$, then we get a ``residue 
seminorm'' on $A/N$, defined by 
\[
  \|g+N\| = \inf\{\|f\|\colon f-g\in N\} .
\]
This gives a norm on $A/N$ if and only if $N$ is closed. 

Let $\varphi\colon M\to N$ be a homomorphism between seminormed groups. We call 
$\varphi$ \emph{bounded} if there exists some $C>0$ such that 
$\|\varphi(f)\|\leqslant C \|f\|$ for all $f\in M$. Clearly bounded functions 
are continuous. The converse does not hold in this level of generality. 

We say two seminorms $\|\cdot\|,\|\cdot\|'$ on a group are \emph{equivalent} if 
there exists $C_1,C_2>0$ such that 
\[
  C_1 \|\cdot\| \leqslant \|\cdot\|' \leqslant C_2 \|\cdot\| .
\]
Equivalent seminorms induce the same topology, but the converse is false in 
general. 

\begin{example}
Let $K$ be any field; consider $M=k[x]$. Let $f=\sum_0^n a_i x^i\in M$, and 
pick some $0<\alpha<1<\alpha'$. Define two seminorms on $M$ by:
\begin{align*}
  \|f\| &= \max\{\alpha^j\colon \alpha_j\ne 0\} \\
  \|f\|' &= \max\{{\alpha'}^j\colon \alpha_j\ne 0\} .
\end{align*}
Let $\varphi\colon (M,\|\cdot\|')\to (M,\|\cdot\|)$ be the identity on $M$. It 
is easy to check that $\varphi$ is bounded. Indeed, 
\[
  \frac{\|\varphi(f)\|}{\|f\|'} = \frac{\|f\|}{\|f\|'} = \frac{\alpha^j}{{\alpha'}^i} \leqslant 1 .
\]
Now $\varphi^{-1}$ is \emph{not} bounded (this is easy), but 
$\varphi^{-1}$ is continuous. We conclude that continuity does not imply 
boundedness. Moreover, bounded homeomorphisms need not have bounded inverses. 
\end{example}

Let $\varphi\colon M\to N$ be a linear map of seminormed groups. As groups, 
$M/\ker\varphi\simeq \image\varphi$, but this isomorphis might not respect the 
seminorms on the two groups. 

\begin{definition}
A linear map $\varphi\colon M\to N$ of seminormed groups is \emph{admissible} 
(or \emph{strict}) if the residue seminorm on $M/\ker\varphi$ and the seminorm 
on $\image\varphi$ are equivalent via the natural isomorphism 
$M/\ker\varphi\simeq \image\varphi$. 
\end{definition}

\begin{definition}
A \emph{Banach ring} is a normed ring which is complete (with respect to the 
norm). 
\end{definition}

\begin{example}
Any ring is with the trivial norm 
\[
  \|f\| = \begin{cases} 1 & \text{if }f\ne 0 \\ 0 & \text{if }f=0 \end{cases}
\]
is a Banach ring. 
\end{example}

\begin{example}
The ring $(\bZ,\|\cdot\|_\infty)$ is a Banach ring. The induced topology is 
discrete, but $\|\cdot\|_\infty$ is very far from being equivalent to the 
trivial norm on $\bZ$. 
\end{example}

\begin{example}
Let $A$ be a Banach ring, $\fa\subset A$ a closed ideal. Then $A/\fa$, with the 
residue norm, is a Banach ring. 
\end{example}

\begin{example}
In the context of the above example, if $\fm\subset A$ is any maximal ideal, 
then $A/\fm$ is a Banach field. Here we implicitly take advantage of 
\autoref{lem:maximal-ideal-closed}. 
\end{example}

\begin{lemma}\label{lem:maximal-ideal-closed}
Let $A$ be a Banach ring, $\fm\subset A$ a maximal ideal. Then $\fm$ is closed. 
\end{lemma}
\begin{proof}
If $\fa\subset A$ is any ideal, then its closure $\closure(\fa)$ is also an 
ideal. Of course, if $\fa$ is a proper ideal, then 
$\fa\cap A^\times=\varnothing$. But $A^\times$ is open. Indeed, if 
$x\in A^\times$, then $\disk^-(x,r)\subset A^\times$ for 
$r=\frac 1 2 \|x^{-1}\|^{-1}$. To see this, let $x+h\in \disk^-(x,r)$; we wish 
to show that $x+h\in A^\times$. We know that $\|h\|<\frac 1 2 \|x^{-1}\|^{-1}$, 
so $\|h x^{-1}\| \leqslant \|h\| \|x^{-1}\| < \frac 1 2$. Write 
$x+h=x(1+h x^{-1})$; the element $1+h x^{-1}$ is invertible by 
\autoref{lem:principle-units-invertible}
\end{proof}

\begin{lemma}\label{lem:principle-units-invertible}
Let $A$ be a Banach ring. Then $\disk^-(1,1)\subset A^\times$. 
\end{lemma}
\begin{proof}
We need to show that if $\|x\|<1$, then $1-x$ is invertible. The sequence 
\[
  a_n = 1 + x + \cdots + x^n ,
\]
is Cauchy because $\|x\|<1$, so it converges. One can easily check that the 
limit is $(1-x)^{-1}$. 
\end{proof}

\begin{example}\label{ex:Tate-algebra}
Let $A$ be a Banach ring with norm $\|\cdot\|$. Choose $r\in \bR^{>0}$. We 
define the ring $A\respow{r^{-1} T}\subset A\pow{T}$ to be the set of power 
series $f=\sum_{i\geqslant 0} a_i T^i$ such that 
$\sum \|a_i\| r^i< \infty$. If $A$ is non-archimedean, we only need to require 
$\|a_i\| r^i \to 0$. We call $A\respow{r^{-1} T}$ a ring of \emph{convergent 
power series}. We define a norm on $A\respow{r^{-1} T}$ by: 
\begin{align*}
  \|f\| &= \sum \|a_i\| r^i && \text{archimedean case} \\
  \|f\| &= \max\{ \|a_i\| r^i\} && \text{non-archimedean case} .
\end{align*}
We could have used the first definition in both cases, but the latter is easier 
to handle. Claim: $(A\respow{r^{-1} T},\|\cdot\|)$ is a Banach ring. 
\end{example}

\begin{example}
Let $\{A_i\}_{i\in I}$ be an arbitrary collection of Banach rings. Then the 
direct product $A=\prod A_i$ can be ``too large'' to be a Banach ring. It 
contains a natural Banach ring, namely the \emph{bounded direct product} 
\[
  \prod^\mathrm{b} A_i = \left\{(f_i)_{i\in I}\colon \sup\|f_i\|<\infty\right\} .
\]
This is a banach ring with respect to the norm 
$\|(f_i)\| = \sup \|f_i\|$. 
\end{example}

[also notation for restricted direct product\ldots]
\[
  \prod^\mathrm{c} A_i = \{(f_i)\colon \lim \|f_i\|\to 0\} .
\]
and the direct sum $\bigoplus_i A_i$. Note that the restricted direct product 
and direct sum will not be unital rings. 





\subsection{Berkovich spectrum of a normed ring}

Let $(A,\|\cdot\|)$ be a normed ring. A seminorm $|\cdot|$ on $A$ is called 
\emph{bounded} (with respect to $\|\cdot\|$) if there exists $c>0$ such that 
$|f|\leqslant c\|f\|$ for all $f\in A$. One also calls bounded seminorms 
\emph{admissible}. If $|\cdot|$ is bounded, then the induced map 
$|\cdot|\colon A\to \bR,f\mapsto |f|$ is continuous, where $\bR$ has its usual 
topology and $A$ has the topology induced by $\|\cdot\|$. 

\begin{definition}
Let $A$ be a commutative unital Banach ring with fixed norm $\|\cdot\|$. The 
\emph{spectrum} of $A$, denoted $\Berkspec(A)$, is the set of bounded 
multiplicative seminorms on $A$. For $x\in \Berkspec(A)$, let $|\cdot|_x$ be 
the corresponding seminorm. We give $\Berkspec(A)$ the weakest topology making 
all maps $x\mapsto |f|_x$ continuous. 
\end{definition}

\begin{theorem}
Let $A$ be a commutative Banach ring. Then 
\begin{enumerate}
\item
$\Berkspec(A)\ne\varnothing$. 

\item
$\Berkspec(A)$ is Hausdorff. 

\item
$\Berkspec(A)$ is compact. 
\end{enumerate}
\end{theorem}
\begin{proof}
1. This is the hardest part of the proof. Let $\fm\subset A$ be a maximal 
ideal; $\fm$ is closed by \autoref{lem:maximal-ideal-closed}. The quotient 
$A/\fm$ is a field, complete with respect to the residue norm. If 
$\Berkspec(A/\fm)\ne\varnothing$, then $\Berkspec(A)\ne\varnothing$. Indeed, 
if $|\cdot|$ is a bounded multiplicative seminorm on $A/\fm$, then 
$|\cdot|\colon A\to \bR$ given by $f\mapsto |f+\fm|$ is a bounded 
multiplicative seminorm on $A$. (This is a special case of functoriality: if 
$f\colon A\to B$ is a bounded homomorphism of Banach rings, there is a natural 
map $f^\ast\colon \Berkspec(B)\to \Berkspec(A)$.) Without loss of generality, 
we may assume that $A$ is a field. 

Let $S$ be the set of all non-zero bounded seminorms on $A$. Given 
$|\cdot|,|\cdot|'\in S$, we put $|\cdot|\leqslant |\cdot|'$ if 
$|f|\leqslant |f|'$ for all $f\in A$. This gives $S$ a partial order. 
\ldots Zorn's Lemma\ldots get a minimal element $|\cdot|$. We claim that 
$|\cdot|$ is multiplicative. To show this, it's enough to show that 
$|f^{-1}| = |f|^{-1}$ for all $f\in A^\times$. For, we already know that 
$|f g|\leqslant |f| |g|$, so just compute:
\[
  |f g|^{-1} = |f^{-1} g^{-1}| \leqslant |f^{-1} |g^{-1}| = |f|^{-1} |g|^{-1} .
\]
We always have $1\leqslant |f| |f^{-1}|$, so assume there exists $f$ with 
$|f|^{-1}<|f^{-1}|$. Let $r=|f^{-1}|^{-1}<|f|$. Consider the 
Tate algebra (from \autoref{ex:Tate-algebra}):
\[
  B = A\respow{r^{-1} T} = \left\{\sum_{i\geqslant 0} a_i T^i\colon \sum |a_i| r^i < \infty\right\} .
\]
Note that $|\cdot|$ extends to $B$ by $|\sum a_i T^i| = \sum |a_i| r^i$. We 
claim that $f-T$ is not invertible in $B$. If it did have an inverse, this 
would be $\sum f^{-i} T^i$, which has norm $\sum |f^{-i}| r^i = \sum 1$, which 
diverges. (See Berkovich's book). Consider the map 
$A\to B/\langle f-T\rangle\ne\varnothing$. This induces a smaller bounded 
seminorm on $A$. So $|\cdot|$ is multiplicative. 

2. For $x\ne y\in \Berkspec(A)$, we want to show there are disjoint open 
neighborhoods of $x$ and $y$. Since $x\ne y$, there exists some $f\in A$ such 
that $|f|_x\ne |f|_y$, say $|f|_x<|f|_y$. Choose $r\in \bR$ with 
$|f|_x<r<|f|_y$. Then $\{z\in \Berkspec(A)\colon |f|_z<r\}$ and 
$\{z\in \Berkspec(A)\colon |f|_z>r\}$ are the desired sets. 

3. Let $C=\prod_{f\in A} [0,\|f\|]$; this is compact by Tychonoff's theorem. 
Note that $\Berkspec(A)\hookrightarrow C$ via $x\mapsto (|f|_x)_{f\in A}$. 
Moreover, $\Berkspec(A)\subset C$ is closed because all the defining properties 
of $\Berkspec(A)$ are closed conditions. Thus $\Berkspec(A)$ is itself 
compact. 
\end{proof}

If $\|\cdot\|$ is already multiplicative, then $\Berkspec(A)\ne\varnothing$ 
because $\|\cdot\|\in \Berkspec(A)$. This is much harder in general. We didn't 
actually need the completeness of $A$ in the above definition. This is because 
any bounded seminorm on $A$ has a unique extension to $\widehat A$. We could 
have replaced $\|\cdot\|$ with any equivalent seminorm and get the same 
$\Berkspec(A)$. 

Let $|\cdot|\colon A\to \bR$ be a power-multiplicative bounded seminorm. Then 
$|f|^n \leqslant c\|f\|^n$; take $n$-th roots and let $n\to \infty$ to realize 
that we may assume $c=1$. So all $x\in \Berkspec(A)$ satisfy 
$|f|_x\leqslant \|f\|$ for all $f\in A$. 

The topology of $\Berkspec(A)$ is very mysterious. In general, one needs model 
theory to prove some basic facts (e.g.~that $\Berkspec(A)$ ``looks like'' a 
simplicial complex). Here are some equivalent ways to define the topology:
\begin{itemize}
\item
The topology on $X=\Berkspec(A)$ is generated by open sets of the form 
\[
  \{x\in X\colon |f|_x<\alpha\} , \qquad \{x\in X\colon |f|_x>\alpha\} ,
\]
for $f\in A$, $\alpha\in \bR$. 

\item
A net $\{x_\alpha\}\subset X$ converges to $x$ if and only if 
$|f|_{x_\alpha} \to |f|_x$ for all $f\in A$. 
\end{itemize}

Let $x\in \Berkspec(A)$. Then $\fp_x = \ker(|\cdot|_x)$ is a closed prime 
ideal. For $f\in A$, $|f|_x$ depends only on $\bar f\in A/\fp_x$. But 
$|\cdot|_x$ is a bounded multiplicative norm on $A/\fp_x$. The ring $A/\fp_x$ 
is an integral domain, so we can pass to its field of fractions 
$(A/\fp_x)_{(0)}$, which carries the norm induced by $|\cdot|_x$. Let 
$\sH(x) = \widehat{(A/\fp_x)_{(0)}}$; this is the \emph{completed residue 
field} of $A$ at $x$. For $f\in A$, write $f(x)$ for the image of $f$ under the 
composite map 
\[
  A \twoheadrightarrow A/\fp_x \hookrightarrow (A/\fp_x)_{(0)} \hookrightarrow \sH(x) .
\]
The map $A\to \sH(x)$ is a ``character,'' in the following sense. 

\begin{definition}
Let $A$ be a Banach ring. A \emph{character} on $A$ is a nonzero bounded 
homomorphism to a field complete with respect to some absolute value. 
\end{definition}
