% !TEX root = 7350.tex

\section{Berkovich spaces}





\subsection{Spectrum of a commutative Banach rings}

This generalizes Gelfand's theory. Much of the theory even works for 
non-commutative algebras. We do not require the existence of a base field, so 
our theory applies to ``arithmetic'' rings like $\bZ$ or 
$\bZ_2[x]/(7 x^3-5)$. 

We start by defining norms and seminorms. (We could phrase everything in terms 
of (rank-one) valuations and semivaluations.) Let $A$ be a commutative (unital) 
ring. 

\begin{definition}
A \emph{seminorm} on $A$ is a function $\|\cdot\|\colon A\to \bR^{\geqslant 0}$ 
satisfying 
\begin{enumerate}
\item
$\|0\|=0$, $\|1\|\leqslant 1$, 

\item
$\|f+g\|\leqslant \|f\|+\|g\|$, 

\item
$\|f g\| \leqslant \|f\| \|g\|$. 
\end{enumerate}
\end{definition}

If we replace axiom 2 with $\|f+g\| \leqslant \max\{\|f\|,\|g\|\}$, we say the 
norm $\|\cdot\|$ is non-archimedean. We will focus primarily on non-archimedean 
norms. We allow $x\ne 0$ to have $\|x\|=0$. Define 
\[
  \ker(\|\cdot\|) = \{x\in A\colon \|x\| = 0\} .
\]
(Huber calls this the \emph{support} of $\|\cdot\|$.) This is a prime ideal in 
$A$. We call $\|\cdot\|$ a \emph{norm} if $\ker(\|\cdot\|)=0$. It turns out 
that our hypotheses imply that either $\|1\|=0$ or $\|1\|=1$. For, if 
$\|1\|\ne 0$, then $\|1\| \leqslant \|1\|\|1\|$, so $1\leqslant \|1\|$. If 
$\|1\|=0$, then $\|f\|=0$ for all $f\in A$. We want to allow this possibility 
so that the zero ring can have a norm. 

If $(A,\|\cdot\|)$ is a ring with seminorm, then $(A,+)$ is naturally a 
seminormed group, i.e.~it satisfies 
\begin{align*}
  \|0\| &=0 \\
  \|f+g\| &\leqslant \|f\|+\|g\| \\
  \|-f\| &= \|f\| .
\end{align*}
(Alternatively, replace the last two equations by 
$\|f-g\| \leqslant \|f\|+\|g\|$.) Any seminormed group carries a natural 
topology, so we get a topology on $A$. This topology is Hausdorff if and only 
if $\ker(\|\cdot\|)=0$. The (Hausdorff) completion $\widehat A$ exists (it can 
be defined via Cauchy sequences as usual), but the obvious map 
$A\to \widehat A$ is injective if and only if $\ker(\|\cdot\|)=0$. 

If we replace the axiom $\|f g\| \leqslant \|f\| \|g\|$ with 
$\|f g\| = \|f\| \|g\|$, we say that $\|\cdot\|$ is \emph{multiplicative}. If 
for all $f\in A$, $\|f^n\| = \|f\|^n$, we say $\|\cdot\|$ is 
\emph{power-multiplicative}. 

If $N$ is a subgroup of a seminormed group $(A,+)$, then we get a ``residue 
seminorm'' on $A/N$, defined by 
\[
  \|g+N\| = \inf\{\|f\|\colon f-g\in N\} .
\]
This gives a norm on $A/N$ if and only if $N$ is closed. 

Let $\varphi\colon M\to N$ be a homomorphism between seminormed groups. We call 
$\varphi$ \emph{bounded} if there exists some $C>0$ such that 
$\|\varphi(f)\|\leqslant C \|f\|$ for all $f\in M$. Clearly bounded functions 
are continuous. The converse does not hold in this level of generality. 

We say two seminorms $\|\cdot\|,\|\cdot\|'$ on a group are \emph{equivalent} if 
there exists $C,C'>0$ such that 
\[
  C' \|\cdot\| \leqslant \|\cdot\|' \leqslant C \|\cdot\| .
\]
Equivalent seminorms induce the same topology, but the converse is false in 
general. 
