% !TEX root = 7350.tex

\section{Foundations}

\subsection{Absolute values and valuations}

A good source for this section is the textbook \cite{engler-prestel}. 
Fix a field $K$ for the remainder of the section. 

\begin{definition}
An \emph{absolute value} (norm) on $K$ is a function $|\cdot|\colon K\to \bR$ 
with the following properties: 
\begin{enumerate}
\item
$|x|=0$ if and only if $x=0$

\item
$|xy|=|x| |y|$

\item $|x+y|\leqslant |x|+|y|$. 
\end{enumerate}
\end{definition}

Eventually, we will relax many of these requirements. The obvious examples 
are $\bR$, $\bC$ with the standard absolute value. The definition has some 
immediate consequences. 

\begin{lemma}
Let $(K,|\cdot|)$ be a field with absolute value. Then 
\begin{enumerate}
\item $|1|=1$
\item $|1/x|=1/|x|$
\item If $x\in \bmu(K)$, then $|x|=1$
\item $|-x|=|x|$
\item $|\cdot|\colon K^\times\to \bR^{>0}$ is a group homomorphism
\end{enumerate}
\end{lemma}

Given an absolute value, we can define a metric on $K$ by 
\[
  d(x,y) = |x-y| ,
\]
hence $K$ becomes a topological space. We call the topology induced by 
$|\cdot|$ the \emph{canonical} topology on $K$. To be pedantic, $U\subset K$ 
is open if for all $u\in U$, there exists $r>0$ such that the ``open disk''
\[
  \disk^-(a,r) = \{x\in K:|x-u|<r\}\subset U .
\]

\begin{lemma}
The topology induced by $|\cdot|$ is discrete if and only if $|\cdot|$ is 
the ``stupid'' absolute value (also known as the trivial absolute value) 
given by 
\[
  |x|=\begin{cases} 1 & x\ne 0 \\ 0 & x=0 \end{cases} .
\]
\end{lemma}
\begin{proof}
Indeed, $\Leftarrow$ is trivial. Showing $\Rightarrow$ is not trivial. If 
$|\cdot|$ is not trivial, then there exists $x\in K$ with 
$0<|x|<1$. The sequence $\{x^n:n\geqslant 1\}$ converges to zero because 
$|x^n|=|x|^n\to 0$. Since we're in a field, no $x^n=0$, so $\{0\}$ is 
\emph{not} open. 
\end{proof}

In fact, we get a topological field: addition, multiplication, and inversion 
are all continuous. Moreover, $K$ is Hausdorff under the canonical topology. 
So far so good. Given an absolute value on a field, we get a Hausdorff topological 
field. However, much of the course will be centered around remedying various 
defects of the canonical topology. 

\begin{lemma}
If $|\cdot|$ is an absolute value on $K$ and $0<e\leqslant 1$, then 
$|\cdot|^e$ is also an absolute value on $K$. 
\end{lemma}
\begin{proof}
$|\cdot|^e$ is trivially multiplicative. Showing the triangle inequality 
is harder, and needs $e\leqslant 1$. It turns out that if $|\cdot|$ is 
non-archimedean, then $e>1$ works as well. 
\end{proof}

\begin{theorem}\label{thm:field-equiv-abs}
Let $|\cdot|$, $|\cdot|'$ be two absolute values on $K$. These induce the 
same topology on $K$ if and only if there exists $e>0$ such that 
$|\cdot|'=|\cdot|^e$. 
\end{theorem}
\begin{proof}
$\Leftarrow$ is obvious. For the converse, see [cite source]. 
\end{proof}

Recall that for any field (commutative ring, even) $K$, there is a unique 
unital ring homomorphism $\bZ\to K$, determined by $1\mapsto 1$. 

\begin{definition}
Let $|\cdot|\colon K\to \bR$ be an absolute value. Let $f:\bZ\to K$ be the 
unique ring map. If $\image(f)$ is bounded in $\bR$, we say $|\cdot|$ is 
\emph{non-archimedean}. 
\end{definition}

This is a topological condition. In other words, if $|\cdot|$ and $|\cdot|'$ 
are equivalent absolute values, then one is non-archimedean if and only if the 
other is. 

\begin{example}
The trivial absolute value is non-archimedean. 
\end{example}

\begin{example}
If $K$ has positive characteristic, all absolute values on $K$ are 
non-archimedean. Indeed, the image of $\bZ\to K$ consists of 
$\{0\}\cup \bF_p^\times$; the latter set consists of $(p-1)$-st roots of 
unity. So $|n|\leqslant 1$ for all $n\in \bZ$. 
\end{example}

\begin{example}
The field $\bQ$ with its usual absolute value $|\cdot|_\infty$ is archimedean. 
\end{example}

\begin{theorem}\label{thm:non-arch-characterization}
An absolute value $|\cdot|$ on $K$ is non-archimedean if and only if for all 
$x,y\in K$, 
\begin{equation}\label{eq:triangle}
  |x+y|\leqslant \max\{|x|,|y|\} .
\end{equation}
\end{theorem}
\begin{proof}
$\Leftarrow$ is easy. Indeed, $|n|\leqslant 1$ by induction on $n$. 

$\Rightarrow$ is more subtle. There exists some $C$ such that 
$|n|\leqslant C$ for all $n\in \bZ$. We use the binomial theorem: 
\begin{align*}
  |(x+y)^n| 
    &= \left|\sum \binom n i x^i y^{n-i} \right| \\
    &\leqslant \sum \left|\binom n i \right| |x^i y^{n-i}| \\
    &\leqslant C(n+1)\max\{|x|,|y|\}^n .
\end{align*}
Taking $n$-th roots of both sides, we obtain 
\[
  |x+y|\leqslant \sqrt[n]{C(n+1)} \max\{|x|,|y|\} .
\]
Letting $n\to \infty$, we obtain the result. 
\end{proof}

The inequality \eqref{eq:triangle} is called the ultrametric inequality in 
older sources. Today, it is generally called the non-archimedean triangle 
inequality. Since the non-archimedean triangle inequality holds when raised to 
arbitrary positive powers, if $|\cdot|$ is non-archimedean, then 
$|\cdot|^e$ is an absolute value for all $e>0$. 

The following simple lemma is quite important. 

\begin{lemma}\label{lem:abs-value-of-difference}
In a non-archimedean field, if $|x|\ne |y|$, then 
$|x+y|=\max\{|x|,|y|\}$. 
\end{lemma}
\begin{proof}
We may assume $|x|<|y|$. We need to show $|x+y|=|y|$. If not, we have 
$|x+y|<|y|$. But 
\[
  |y|=|y+x-x| \leqslant \max\{|y+x|,|x|\} < y ,
\]
a contradiction. 
\end{proof}





\subsection{Topological consequences}

\subsubsection{Center of a disk}

Let $(K,|\cdot|)$ be a non-archimedean field. Let $a,a'\in K$ with 
$|a-a'|\leqslant r$ for some $r>0$. Then $|x-a|\leqslant r$ if and only if 
$|x-a'|\leqslant r$. In other words: \emph{all points in a disk of radius $r$ 
are the center of the disk}! In other words, any point in the set  
\[
	\disk(a,r) = \{x\in K:|x-a|\leqslant r\} ,
\]
is a ``center'' of $\disk(a,r)$. To see this, suppose $|x-a|\leqslant r$. Then 
\begin{align*}
  |x-a'| 
    &= |x-a+a-a'| \\
    &\leqslant \max\{|x-a|,|a-a'|\} \\
    &\leqslant r .
\end{align*}

\subsubsection{Clopen sets}
Clearly the ``closed disk'' $\disk(a,r)$ is closed with respect to the canonical 
topology. (This works for any metric space.) The terrible problem with 
non-archimedean fields is that $\disk(a,r)$ is also open! Indeed, let 
$x_0\in \disk(a,r)$ (so $|x_0-a|\leqslant r$) and consider the ``open ball'' 
\[
  \disk^-(x_0,r) = \{x\in K:|x-x_0|<r\} .
\]
We have $\disk^-(x_0,r)\subset \disk(a,r)$. 

\subsubsection{Boundaries}
Contrary to intuition, in a non-archimedean field, the set 
$\{x\colon |x-a|=r\}$ is \emph{not} the topological boundary (closure 
$\smallsetminus$ interior) of $\disk(a,r)$. In fact, any small neighborhood of 
a point in this set is contained in $\disk(a,r)$. 

\subsubsection{Disconnectedness}
Putting everything together, the canonical topology on $K$ is totally 
disconnected. In other words, the only non-empty connected components are 
singletons. Indeed, let $x_0\in K$, and let $A$ be the connected component 
containing $x_0$. Assume $x_1\ne x_0$ is also in $A$. Let 
$|x_1-x_0|>r>0$. Then 
\[
  A = \left(A\cap \disk(r,x_0)\right) \sqcup \left(A\smallsetminus \disk(r,x_0)\right) 
\]
is a decomposition of $A$ into a disjoint union of open nonempty subsets. 





\subsection{Ostrowski's theorem}

Consider the field $\bQ$. We already know of the following absolute values: 
\begin{enumerate}
\item
$|\cdot|_\infty^e$ for any $0<e\leqslant 1$.

\item 
$|\cdot|_0$, the trivial absolute value.

\item 
For each prime $p$, we have an absolute value $|\cdot|_p^e$ for any $e>0$. Recall 
that for any $\frac a b\in \bQ$, write $\frac a b=p^\alpha \frac{a'}{b'}$ for 
$a',b'$ not divisible by $p$. One has $|a/b|_p^e = (p^e)^{-\alpha}$. We 
generally normalize by requiring $|p|_p = 1/p$; there are good analytical 
reasons from this involving normalization of Haar measures. 
\end{enumerate}

The field $\bQ$ is totally disconnected for all these topologies. For 
$|\cdot|_\infty$, this can be solved via completion. For the $p$-adic and 
trivial absolute values, we need Berkovich's theory. 

\begin{theorem}[Ostrowski]\label{thm:ostrowski}
All absolute values on $\bQ$ are, up to topological equivalence, of the form 
$|\cdot|_0$, $|\cdot|_\infty$, or $|\cdot|_p$ for some $p$. 
\end{theorem}
\begin{proof}
Let $|\cdot|$ be an absolute value on $\bQ$. 

Case 1: $|\cdot|$ is non-archimedean. Then $|n|\leqslant 1$ for all $n\in \bZ$. 
If $|p|=1$ for all primes $p$, then by unique factorization, $|\cdot|$ is the 
trivial absolute value. If there exists a prime $p$ for which $|p|<1$, there is 
no $l\ne p$ prime such that $|l|<1$. For, since $l$ and $p$ are coprime, there 
is $x,y\in \bZ$ for which $p x+l y=1$. But then 
\[
  1 = |1| = |p x+l y| \leqslant \max\{|p x|,|p y|\} < 1 ,
\]
a contradiction. So $|p|<1$ for a unique $p$. After possibly replacing 
$|\cdot|$ by an equivalent absolute value, we may assume $|p|=1/p$. Unique 
factorization, combined with $|l|=1$ ($l\ne p$) tells us that 
$|\cdot|=|\cdot|_p$. 

Case 2: $|\cdot|$ is archimedean. There exists $n\in \bZ^{>0}$ such that 
$|n|>1$; let $n_0$ be the smallest such $n$. Write $|n_0|=n_0^\alpha$. Given 
$n\in \bZ^{>0}$, we need to show $|n|=|n|_\infty^\alpha$. Write $n$ in base $n_0$: 
\begin{equation}\label{eq:base-n}
  n = a_0 + a_1 n_0 + \cdots + a_s n_0^s ,
\end{equation}
where each $0\leqslant a_0<n_0$ and $a_s\ne 0$. First we show 
$|n|\leqslant |n|_\infty^\alpha$. Put $C=\sum_{i\geqslant 0} n_0^{-\alpha i}$; 
then 
\begin{align*}
  |n| 
    &= \bigg|\sum_{i=0}^s a_i n_0^i\bigg| \\
    &\leqslant \sum_{i=0}^s |a_i| |n_0|^i \\
    &\leqslant \sum_{i=0}^s n_0^{\alpha i} \\
    &\leqslant n_0^{s\alpha} (1+n_0^{-\alpha} + n_0^{-2\alpha} + \cdots) \\
    &\leqslant C n^\alpha .
\end{align*}
So $|n^N|\leqslant C n^{\alpha N}$ for all $N$. Thus 
$|n|\leqslant \sqrt[N]{C} n^\alpha$; letting $N\to \infty$ yields 
$|n|\leqslant n^\alpha$. Now we prove $|n|\geqslant n^\alpha$. Once again 
write $n$ in base $n_0$ as in \eqref{eq:base-n}. We get 
\begin{align*}
  |n_0^{s+1}| 
    &\leqslant |n|+|n_0^{s+1}-n| \\
    &\leqslant |n|+(n_0^{s+1}-n)^\alpha \\
    &\leqslant |n|+(n_0^{s+1}-n_0^s)^\alpha .
\end{align*}
It follows that
\[
  |n| 
    \geqslant n_0^{\alpha(s+1)} \left(1-\left(1-\frac{1}{n_0}\right)^\alpha\right) \\
    \geqslant C' n^\alpha ,
\]
where as above $C'$ does not depend on $n$. The same trick (raise to $N$, let 
$N\to \infty$) yields the result. 
\end{proof}

There are versions of Ostrowski's theorem for arbitrary Dedekind domains, or 
function fields like $\bF_p(t)$. 





\subsection{Completion}\label{sec:completion}

Here, will discuss ``completion'' in the sense of metric spaces. One can also 
``complete'' topological spaces; there is a more general notion of completion 
that works for ``uniform spaces,'' which are topological spaces together with a 
uniform structure. We will focus on metrics coming from absolute values. 

\begin{definition}
A metric space is \emph{complete} if every Cauchy sequence converges. 
\end{definition}

Let $K$ be a field with absolute value $|\cdot|$. Note that $(K,|\cdot|)$ is 
complete if and only if $(K,|\cdot|^e)$ is complete, so completeness of a field 
is a topological property. 

\begin{example}
The field $\bQ$ with $|\cdot|_0$ is complete (Cauchy sequences stabilize). 
\end{example}

\begin{example}
The field $\bQ$ with $|\cdot|_\infty$ is \emph{not} complete. It's completion 
is denoted $\bR$; this is as nice as we could hope for (as a topological 
space). 
\end{example}

\begin{example}
The field $\bQ$ with $|\cdot|_p$ is also not complete. For example, for any 
integer $n\geqslant 1$, the sequence 
\[
  n,n^p,n^{p^2},n^{p^3},\dots
\]
is Cauchy, and (if $p$ is odd) it does not converge in $\bQ$ when we set 
$n=p-1$. Alternatively, it is known that any complete metric space without 
isolated points is uncountable; since $\bQ$ is countable, it cannot be complete 
in any non-discrete topology. 
\end{example}

\begin{definition}
Let $(K,|\cdot|)$ be a field with absolute value. A \emph{completion} of 
$(K,|\cdot|)$ is tuple $(K',|\cdot|',i)$, where 
\begin{enumerate}
\item $(K',|\cdot|')$ is a complete field, 
\item $i\colon K\hookrightarrow K'$ is an isometry 
(i.e.~$|x|=|i(x)|'$ for all $x\in K$), and 
\item for any isometry $j\colon K\hookrightarrow F$ from $K$ into a complete field 
$(F,|\cdot|'')$, there exists a unique isometry $i'\colon K'\hookrightarrow F$ 
such that the following diagram commutes:
\[
\begin{tikzcd}
  K \ar[r, "i"] \ar[dr, "j"]
    & K' \ar[d, "i'", dotted] \\
  & F .
\end{tikzcd}
\]
\end{enumerate}
\end{definition}

So a completion is an embedding of $K$ into a complete field that is 
``initial'' among such embeddings. Standard arguments show that the completion 
of $K$ is uniquely determined by the universal property. 

\begin{theorem}\label{thm:existence-completion}
Let $K$ be a field with absolute value $|\cdot|$. Then: 
\begin{enumerate}
\item A completion $(K',|\cdot|',i)$ of $K$ exists.  
\item $i(K)$ is dense in $K'$.
\item $|\cdot|$ is non-archimedean if and only if $|\cdot|'$ is non-archimedean.  
\item If $|\cdot|$ is non-archimedean, then 
\[
  |K^\times| = \{|x|\colon x\in K^\times\} = |{K'}^\times|' .
\]
\end{enumerate}
\end{theorem}
\begin{proof}
1, 2 are classical. Construct $K'$ as the quotient of the ring of Cauchy 
sequences by the (maximal) ideal of nullsequences. 

3. Obvious corollary of \autoref{thm:non-arch-characterization}. 

4. Roughly: suppose $x\in K'$ is written $x=\lim x_n$, for $\{x_n\}$ a Cauchy 
sequence in $K$. Then $|x|'=\lim |x_n|$. For $m,n\gg 0$, we have 
$|x_n-x_m|<|x|/2<|x_n|$. By \autoref{lem:abs-value-of-difference}, this can 
only happen if $|x_n|=|x_m|$. So the sequence $\{|x_n|\}$ stabilizes. 
\end{proof}

It is true (though we will not prove) that if $K\hookrightarrow K'$ is 
\emph{any} isometry from $K$ to a complete field with dense image, then $K'$ is 
a completion of $K$. 

Part 4 clearly fails when passing from $\bQ$ to $\bR$. On the other hand, now 
we know that $\bQ_p$ has ``no new absolute values.'' 





\subsection{The \texorpdfstring{$p$}{p}-adics}

For each prime $p$, we write $\bQ_p$ for any completion of $\bQ$ with respect 
to $|\cdot|_p$; we call $\bQ_p$ the \emph{field of $p$-adic numbers}. Define 
\begin{align*}
  \bQ_p^\circ = \bZ_p &= \{x\in \bQ_p\colon |x|\leqslant 1\} \\
  \bQ_p^{\circ\circ} = \fm &= \{x\in \bQ_p\colon |x|<1\} .
\end{align*}
One calls $\bZ_p$ the ring of \emph{$p$-adic integers}. One has 
$\bZ_p/\fm\simeq \bF_p$. Choose a 
set $S=\{\alpha_0,\dots,\alpha_{p-1}\}$, where $\alpha_i$ is a $p$-adic integer 
such that $\alpha_i\equiv i\pmod p$. We could use $S=\{0,\dots,p-1\}$, but 
there are better choices (e.g. Teichm\"uller representatives). Then some 
elementary work shows that every $\alpha\in \bQ_p$ has exactly one 
representative Cauchy sequence $\{a_i\}$, where 
\[
  a_i = b_0 p^{-m} + b_1 p^{-(m-1)} + \cdots + b_{i-1} p^{i-1-m} ,
\]
where the $b_j\in S$. 

\begin{example}
Work in $\bQ_3$ and let $S=\{0,1,2\}$. One such sequence is: 
\begin{align*}
  a_1 &= 1 \\
  a_2 &= 1+0\cdot 3 \\
  a_3 &= 1+0\cdot 3+2\cdot 3^2 \\
  a_4 &= 1+0\cdot 3+3\cdot 3^2 + 1\cdot 3^3 \\
  \vdots &
\end{align*}
\end{example}

Given a representation 
\[
  a = b_0 p^{-m} + b_1 p^{-(m-1)} + \cdots ,
\]
where $b_0\ne 0$, one can check that $|a|_p = p^m$. 

The field $\bQ_p$ is complete, totally disconnected, but not algebraically 
closed. An abstract algebraic closure $\overline{\bQ_p}$ exists. However, 
unlike the passage from $\bR$ to $\bC$, the extension $\overline{\bQ_p}/\bQ_p$ 
is \emph{not} finite. Moreover, $\overline{\bQ_p}$ is not complete with respect 
to the unique absolute value extending the one on $\bQ_p$. Let 
$\bC_p=\widehat{\overline{\bQ_p}}$; here we we are lucky and $\bC_p$ is 
algebraically closed. The field $\bC_p$ is the $p$-adic analogue of $\bC$. 





\subsection{Absolute values under algebraic extensions}

A field extension is naturally a vector space over the base field, so our 
first results will be in the more general context of vector spaces over 
normed fields. 

\begin{definition}
Let $K$ be a field with absolute value $|\cdot|$, $V$ be a vector space over 
$K$. A \emph{($K$-vector space) norm} on $V$ is a map 
$\|\cdot\|\colon V\to \bR$, satisfying: 
\begin{enumerate}
\item 
$\|v\|\geqslant 0$ for all $v\in V$, with equality if and only if $v=0$. 

\item
$\|c v\| = |c|\|v\|$ for all $c\in K$, $v\in V$. 

\item
$\|v+w\|\leqslant \|v\|+\|w\|$ for all $v,w\in V$. 
\end{enumerate}
\end{definition}

We get in the natural way a metric (hence topology) on $V$. Thus concepts like 
completeness and completion make sense. The whole machinery of 
\autoref{sec:completion} carries through. 

Two norms $\|\cdot\|_1$, $\|\cdot\|_2$ induce some topology if and only if 
\[
  c_2 \|\cdot\|_2 \leqslant \|\cdot\|_1 \leqslant c1 \|\cdot\|_2 ,
\]
for some $c_1,c_2>0$. We call such norms \emph{equivalent}. 
In classical linear algebra, one often rescales vectors 
to make them have norm one. Here, if $v\in V$, we might not be able to 
rescale $v$ to get $\|cv\|=1$. This is because $\{|c|\colon c\in K^\times\}$ 
can be a proper subgroup of $\bR^{>0}$. 

\begin{example}
Fix a basis $B=\{v_1,\dots,v_d\}$ of $V$. Given $x=\sum a_i v_i$, put 
\[
  \|x\|_{\sup,B} = \|x\|_{\max,B} = \max_{1\leqslant i\leqslant d} |a_i| .
\]
It is easy to show that this actually is a norm. If $K$ is complete with 
respect to $|\cdot|$, then $(V,\|\cdot\|_{\max})$ is also complete. If 
$|\cdot|$ is non-archimedean, then 
\[
  \|v+w\|_{\max} \leqslant \max\{\|v\|_{\max},\|w\|_{\max}\} .
\]
Finally, the max norm with respect to different bases are equivalent. 
\end{example}

\begin{theorem}\label{thm:unique-norm-vect}
If $K$ is complete with respect to $|\cdot|$ and $V$ is finite-dimensional over 
$K$, then all $K$-vector space norms on $V$ are equivalent. Moreover, $V$ is 
also complete, all vector subspaces are closed, and if $|\cdot|$ is 
non-archimedean, then $\|\cdot\|$ is also non-archimedean, in the weaker sense 
that 
\[
  \|v+w\| \leqslant C\max\{\|v\|,\|w\|\} ,
\]
for some fixed $C$. 
\end{theorem}
\begin{proof}
Fix a basis $B=\{v_1,\dots,v_d\}$ of $V$. Given any norm $\|\cdot\|$ on $V$, we 
show it's equivalent to $\|\cdot\|_{\max,B}$. One direction doesn't use 
completeness: 
\begin{align*}
  \|x\| 
    &= \left\|\sum a_i v_i\right\| \\
    &\leqslant \sum |a_i| \|v_i\| \\
    &\leqslant d \max_i \|v_i\| \max_ii |a_i| .
\end{align*}
Put $C_1=d \max\{v_i\}$; this yields 
$|\cdot|\leqslant C_1 \|\cdot\|_{\max,B}$. To show that 
$\|\cdot\|_{\max}\leqslant C_2\|\cdot\|$, we induct on $d=\dim(V)$. For $d=1$, 
the result is immediate: just observe $\|v\|=|a_1|\|v_1\|$. 
If $d>1$, we may assume that all proper subspaces of $V$ are complete. General 
nonsense tells us that subspaces are closed. Let $H_i=\sum_{j\ne i} K v_j$. 
Let $\pi_i\colon V\to V/H_i = L_i$ be the projection map. Define a new norm on $L_i$ 
by 
\[
  \|u\|_i = \inf\{\|v\|\colon \pi_i(v)=u\} .
\]
Since $H_i$ is closed, $\|u_i\|=0$ implies $u\in H_i$. For $v=\sum a_i v_i$, 
we have $\pi_i(v)=a_i \pi_v(v_i)$. Note that 
$\|\pi_i(v)\|_i = |a_i| \|\pi_i(v_i)\|_i$, which implies 
\[
  \frac{\|\pi_i(v)\|_i}{\|\pi_i(v_i)\|_i} \leqslant \frac{\|v\|}{\|\pi_i(v_i)\|_i} \leqslant B_i \|v\| ,
\]
where $B_i=\|\pi_i(v_i)\|_i^{-1}$. Put $C_2=\max\{B_i\}$. 
\end{proof}

The moral of the story is that finite-dimensional vector spaces over a complete 
field carry a unique topology. 

\begin{theorem}\label{thm:extend-absolute-value}
Let $K$ be a field complete with respect to $|\cdot|$. Let $L/K$ be an 
extension. 
\begin{enumerate}
\item 
There is exactly one extension of $|\cdot|$ to an absolute value $|\cdot|'$ on 
$L$. Moreover, $|\cdot|'$ is non-archimedean if and only if $|\cdot|$ is 
non-archimedean. 

\item
If $[L \colon K]<\infty$, then $(L,|\cdot|')$ is complete. If $K$ is 
non-archimedean, then for all $x\in L$, 
\begin{equation}\label{eq:extend-norm}
  |x|' = \left|\norm_{L/K}(x)\right|^{\frac{1}{[L\colon K]}} .
\end{equation}
\end{enumerate}
\end{theorem}
\begin{proof}
1. That there is \emph{at most} one extension follows almost immediately from 
\autoref{thm:unique-norm-vect}. The field $L$ is a direct limit of finite 
extensions, so we may as well assume $L$ is a finite extension of $K$. Since 
$L$ is a finite-dimensional $K$-vector space, it has a unique topology 
induced by vector space norm. Since $K$ is complete, $L$ is also complete. Let 
$|\cdot|_1$ and $|\cdot|_2$ be two norms on $L$. If the topology induced 
by $|\cdot|_1$, then everything in sight is discrete. If $|\cdot|_1$ induces 
a non-discrete topology, then by \autoref{thm:field-equiv-abs}, 
$|\cdot|_1=|\cdot|_2^e$ for some $e$. Restricting to $K$, we see that $e=1$. 

2. Recall that for $x\in L$, the map ``multiply by $x$'' is a $K$-linear map 
$(x\cdot)\colon L\to L$. The \emph{norm} of $x$ is by definition 
$\norm_{L/K}(x)=\det(x\cdot)$. It's not easy to show that 
\eqref{eq:extend-norm} satisfies the triangle inequality. (One uses Hensel's 
lemma to show that $\norm_{L/K}(x)\in K^\circ$ if and only if 
$x\in L^\circ$.)
\end{proof}

It is easy to show that \eqref{eq:extend-norm} is the only possible extension 
of a norm, using the Galois definition of the norm. Abstract nonsense reduces 
us to the case where $L/K$ is finite Galois. In this case, we know that 
\[
  \norm_{L/K}(x) = \prod_{\sigma\in \galois(L/K)}\sigma(x) .
\]
It follows that 
\begin{align*}
  |\norm_{L/K}(x)| 
    &= |\norm_{L/K}(x)|' \\
    &= \prod_{\sigma\in \galois(L/K)} |\sigma(x)|' \\
    &= {|x|'}^{[L\colon K]} .
\end{align*}
The last equality holds because $|\sigma(x)|'=|x|'$ for all $\sigma$. 

\begin{example}
Let $\bQ_p$ be the completion of $\bQ$ with respect to the $p$-adic absolute 
value $|\cdot|_p$. By \autoref{thm:extend-absolute-value}, 
$\overline{\bQ_p}$ carries a unique extension of $|\cdot|_p$; the set 
$|\overline{\bQ_p}^\times|\subset \bR^{>0}$ is dense!
\end{example}

In general, if $K$ is a complete non-archimedean field, then its algebraic 
closure $\overline K$ will not be complete. The only exception is when 
$\overline K=\bC$.

\begin{theorem}[Gelfand, Mazur]\label{thm:Gelfand-Mazur}
The only complete archimedean fields are $(\bR,|\cdot|_\infty^e)$ and 
$(\bC,|\cdot|_\infty^e)$ for $0\leqslant e<1$. 
\end{theorem}
\begin{proof}
This is \cite[1.2.4]{engler-prestel}. If $(K,|\cdot|)$ is a complete 
archimedean field, then $K$ has characteristic zero, so $\bQ\hookrightarrow K$. 
The absolute value $|\cdot|$ restricted to $\bQ$ is archimedean, so by 
\autoref{thm:ostrowski}, $|\cdot|=|\cdot|_\infty^e$ on $\bQ$ for some 
$0<e\leqslant 1$. By the universal property of (metric) completion, we get a 
continuous embedding $\bR\hookrightarrow K$. By \autoref{cor:exts-of-R}, 
$K=\bR$ or $K=\bC$. 
\end{proof}

\begin{corollary}
If $(K,|\cdot|)$ is an archimedean field, then $K$ admits a dense isometry 
into either $\bR$ or $\bC$. 
\end{corollary}

\begin{lemma}\label{lemma:spec-Banach}
Let $A$ be a commutative Banach $\bR$-algebra containing $\bC$. For every 
$a\in A$, the set $\{c\in \bC\colon a-c\notin A^\times\}$ is compact and 
nonempty. 
\end{lemma}
\begin{proof}
To show that the set, known as the \emph{spectrum} of $a$, is nonempty, 
one uses the map $\bC\to A$ given by $c\mapsto (a-c)^{-1}$. For details, see 
\cite[18.6]{rudin-1987}. 
\end{proof}

\begin{corollary}\label{cor:exts-of-R}
Let $(K,|\cdot|)$ be an extension of $(\bR,|\cdot|_\infty)$. Then either 
$K=\bR$ or $K=\bC$. 
\end{corollary}
\begin{proof}
If $K$ contains $\bC$, then applying \autoref{lemma:spec-Banach} shows that 
$K=\bC$. If $K$ does not contain any such $j$, the field $K(\sqrt{-1})$ must be 
$\bC$, hence $K=\bR$. 
\end{proof}

If $(K,|\cdot|)$ is a non-archimedean field which is \emph{not} complete, and 
$L/K$ is an algebraic extension, how can we extend $|\cdot|$ to $L$? This is 
easy. Let $\widehat K$ be the completion of $K$; it carries a unique extension 
of $|\cdot|$. Let $\overline{\widehat K}$ be an algebraic closure of 
$\widehat K$; this also carries a unique extension of $|\cdot|$. We have an 
embedding $i:L\hookrightarrow \overline{\widehat K}$. The pullback via $i$ of 
the absolute value on $\overline{\widehat K}$ is an extension of $|\cdot|$ to 
$L$. 

\begin{theorem}[Krasner]\label{thm:krasner}
Let $K$ be an algebraically closed field with absolute value $|\cdot|$. Then 
$\widehat K$ is also algebraically closed. 
\end{theorem}
\begin{proof}
This works via ``continuity of roots.'' Let $L$ be an algebraic closure of 
$\widehat K$. We want to prove that $L=\widehat K$. Let $f\in \widehat K[x]$; 
write $f=\sum_{i=0}^n c_i x^i$ with $c_n=1$. Let $\alpha\in L$ be a root of 
$f$, i.e.~$f(\alpha)=0$. It suffices to show that $\alpha$ can be approximated 
by elements $\beta_i\in K$ (so $\alpha=\lim \beta_i\in \widehat K$). Fix 
$\epsilon>0$. We may choose $d_i\in K$ approximating the coefficients 
$c_i\in \widehat K$, so that $g=\sum d_i x^i\in K[x]$ is monic and satisfies 
$|g(\alpha)|\leqslant \epsilon^n$. Since $K$ is algebraically closed, we have a 
factorization $g=\prod (x-\beta_i)$, so 
\[
  |g(\alpha)| = \prod_{i=1}^n |\alpha-\beta_i| \leqslant \epsilon^n .
\]
So at least one $\beta_i$ satisfies $|\alpha-\beta_i|\leqslant \epsilon$. 
\end{proof}

We mentioned that $(\bQ,|\cdot|_p)$ is not complete. There is an explicit, 
algorithmic way to show this, which comes from sequences $\{a_i\mod{p^i}\}$. 
This is done very nicely in [Gouvea, p.~52]. 

For any non-archimedean field $K$, we can pass to the completion $\widehat K$, 
its algebraic closure $\overline{\widehat K}$, then finally to the completion 
of that field: $\widehat{\overline{\widehat K}}$. By \autoref{thm:krasner}, 
this is a complete, algebraically closed, non-archimedean field. The 
absolute value $|\cdot|$ on $K$ has a unique extension to 
$\widehat{\overline{\widehat K}}$. One thing to be careful of: while 
$|K^\times|=|\widehat K^\times|$ and similarly for the embedding 
$\overline{\widehat K}\hookrightarrow\widehat{\overline{\widehat K}}$. But 
$|\overline{\widehat K}^\times|$ may be strictly larger than 
$|\widehat K^\times|$. 

\begin{example}
For $(\bQ,|\cdot|_p)$, we get the following chain of embeddings:
\[
  \bQ\hookrightarrow \bQ_p \hookrightarrow \overline{\bQ_p} \hookrightarrow \widehat{\overline{\bQ_p}} = \bC_p .
\]
The field $\bC_p$ is the ``$p$-adic analogue of $\bC$.'' See 
[Koblitz, $p$-adic\ldots] for a careful explanation of each step. 
\end{example}

\begin{definition}
Let $K$ be any field. The \emph{field of Puiseaux series} over $K$, is 
\[
  K\puiseaux{t} = \left\{\sum_{k=k_0}^\infty a_i t^{k/n}\colon n\in \bZ^{>0}\right\} .
\]
\end{definition}

\begin{theorem}[Newton]
If $K$ is algebraically closed of characteristic zero, then 
$K\puiseaux{t} = \overline{K\laurent{t}}$. 
\end{theorem}

Newton proved this theorem by creating an early version of what is now called 
\emph{Newton's method}. Thus we have a sequence of embeddings 
\[
  K(t)\hookrightarrow 
  K\laurent{t} \hookrightarrow 
  K\puiseaux{t} \hookrightarrow 
  \widehat{K{\puiseaux{t}}} ,
\]
in which the last field is known as the field of \emph{formal Puiseaux 
series}. [Eisenbud's commutative algebra, cor.~13.15]

\begin{definition}
Let $(K,|\cdot|)$ be a non-archimedean field. Put 
\begin{align*}
  K^\circ &= \{x\in K\colon |x|\leqslant 1\} && \text{``valuation ring''} \\
  K^{\circ\circ} &= \{x\in K\colon |x|<1\} && \text{``(unique) maximal ideal''} \\
  K^\natural &= K^\circ / K^{\circ\circ} && \text{``residue (class) field''} .
\end{align*}
\end{definition}

\begin{lemma}
Let $(K,|\cdot|)$ be a non-archimedean field. Then the embedding 
$K\hookrightarrow \widehat K$ induces an isomorphism 
$K^\natural\xrightarrow\sim\widehat K^\natural$.
\end{lemma}
\begin{proof}
It is trivial to check that: 
\begin{align*}
  K^\circ &= \widehat K^\circ\cap K \\
  K^{\circ\circ} &= \widehat K^{\circ\circ}\cap K .
\end{align*}
This gives us a well-defined map $K^\circ\to \widehat K^\natural$, namely 
$a\mapsto a+\widehat K^{\circ\circ}$. It has kernel $K^{\circ\circ}$, so we get 
an embedding $K^\natural\hookrightarrow \widehat K^\natural$. To see that this 
map is surjective, note that for any $x\in \widehat K^\circ$, the set 
$x+\widehat K^{\circ\circ}$ is an open set in $\widehat K$. Since $K$ is dense 
in $\widehat K$, there is some $y\in (x+\widehat K^{\circ\circ})\cap K$; then 
$y\mapsto x+\widehat K^{\circ\circ}$. 
\end{proof}

\begin{definition}
Let $L/K$ be an arbitrary extension of fields with absolute value. If 
$|K^\times|=|L^\times|$ and $K^\natural\xrightarrow\sim L^\natural$, we say $L$ 
is an \emph{immediate extension} of $K$.  
\end{definition}

\begin{example}
If $K$ is any non-archimedean field, then $\widehat K$ is an immediate 
extension of $K$. 
\end{example}

\begin{definition}
We say a field $(K,|\cdot|)$ is \emph{maximally complete} if it admits no 
proper immediate extensions. 
\end{definition}

\begin{definition}
Let $(V,|\cdot|)$ be a non-archimedean normed space. We say $V$ is 
\emph{spherically complete} if, whenever $\{D_i\}$ is a nested set of 
closed disks $\disk(a,r)=\{x\in V\colon |x-a|\leqslant r\}$, then 
$\bigcap_i D_i\ne \varnothing$. 
\end{definition}

It turns out that spherical completeness is important for functional analysis. 
For example, the proof of the Hahn-Banach theorem requires spherical 
completeness. As an easy exercise, show that spherical completeness implies 
completeness. 

\begin{example}
The fields $(\bR,|\cdot|_\infty)$ and $(\bC,|\cdot|_\infty)$ are spherically 
complete. 
\end{example}

\begin{example}
The field $(\bC_p,|\cdot|_p)$ is \emph{not} spherically complete. This is 
non-trivial, and highly frustrating. See [Roberts, ``a course in $p$-adic 
analysis., ch.~3]
\end{example}

\begin{theorem}
A non-archimedean field is spherically complete if and only if it is 
maximally complete. 
\end{theorem}

For a good general discussion of spherically complete fields, see chapter 2 
of ``functional analysis and valuation theory,'' by Narici+\ldots. This 
theorem is proved on p.~34,43.

\begin{theorem}[Krull]
All valued fields have a maximally complete extension. 
\end{theorem}

[previous book, p.~54]

\begin{corollary}
All non-archimedean fields can be embedded into a spherically complete 
extension with the same value group and residue field. 
\end{corollary}

\begin{example}
Let's apply the above results to $\bC_p$. We get an algebraically closed, 
spherically complete extension of $\bQ_p$. Denote this by ???
\end{example}

We would like some notion of uniqueness for maximally complete extensions. 
Sadly, this doesn't work in general. 

\begin{theorem}[Kaplansky]
If $K^\natural$ is characteristic zero, there is a unique maximal extension. 
\end{theorem}

[See Poonen, Hahn series. Also ch.3 of Robert's book, where the 
construction uses ultraproducts.]

Get a field $\Omega_p$ satisfying:
\begin{enumerate}
\item
the field contains $\bQ_p$, $\bC_p$

\item
it's algebraically closed, 

\item
it's spherically complete
\end{enumerate}

Moreover, $|\Omega_p|=\bR^{\geqslant 0}$, as opposed to 
$|\bC_p^\times|=p^\bQ$. As a field, $\bC\simeq \bC_p$ [see logic stuff, 
see \S 3.5 of Robert's book.]




